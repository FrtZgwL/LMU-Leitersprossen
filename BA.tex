% Grundeinstellungen, Formatierung, Ränder, Schrift, Zitierstil, usw.
\documentclass[fontsize=12pt, paper=a4, headings=standardclasses, parskip=half, bibliography=numbered]{scrartcl}
\usepackage[left=3cm,right=3cm,top=3cm,bottom=3cm]{geometry}
\usepackage[rm]{libertine}
\usepackage[english,german,main=ngerman]{babel}
\usepackage[autostyle=true,german=quotes,autopunct=true]{csquotes}
\usepackage[backend=biber,style=authoryear,autocite=footnote]{biblatex}
% \addbibresource{C:\\Users\\Cedric\\Zotero\\My Library.bib}
% windows \addbibresource{/Users/cedrictonyedjon/OneDrive - bwedu/Zotero/Meine Bibliothek.bib}
\addbibresource{/home/cedric/Zotero/My Library.bib}
\deffootnote{1.5em}{1em}{%
  \makebox[1.5em][l]{\thefootnotemark}%
}
\usepackage{fnpct}
\AtBeginEnvironment{quote}{\small}

\makeatletter
\newcommand{\claim}{%
  \@startsection{paragraph}{4}%
  {\z@}{0em \@plus 1ex \@minus .2ex}{-1em}%
  {\normalfont\normalsize\bfseries}%
}
\makeatother

  % Gerstorfers Zeug, von dem ich nicht weiß, wozu es gut ist
\usepackage[unicode=true,pdfborder={0 0 0},hyperfootnotes=false]{hyperref}
\usepackage{bookmark}
\usepackage[onehalfspacing]{setspace}
\usepackage{hyperref}
\usepackage{graphicx}
\usepackage{grffile} % Löst Probleme mit Dateinamen beim Lesen von Bildern
\usepackage{cleveref}

\begin{document}

% Deckblatt
\begin{titlepage}
Bachelorarbeit\\
Universität Stuttgart\\
Institut für Philosophie\\
Apl. Prof. Dr. Jakob Steinbrenner\\
PD Dr. Joachim Bromand\\\\
54,5 reine Textseiten\\
Gefordert waren maximal 55 Textseiten\\
Abgegeben am 31.03.2021

\vfill

\Huge Wittgenstein und die Grenzen\\ unserer Sprache\\\\
\large Lässt sich eine Grenze unserer Sprache im Stil von Wittgensteins\\Tractatus verstehen?
\normalsize

\vfill

Cedric Tonye-Djon\\
Matrikelnummer: 3224391\\
cedrictd.ctd@gmail.com\\
1-Fach-Bachelor Philosophie\\
9. Fachsemester
\end{titlepage}

% Für wen schreibe ich?
% =====================
% Fachwelt!

%Inhaltsverzeichnis
\tableofcontents
\clearpage

\listoffigures
\clearpage

% TODO: Es wär so cool, hier Frege-Style Stichwortartige Zusammenfassungen zu haben

\section{Einleitung}

Manchmal hören wir Sätze wie

\begin{itemize}
  \item Das war so besonders, es lässt sich nicht sprachlich beschreiben.
  \item Mir fehlen die Worte, um dir zu sagen was ich fühle.\footnote{Angelehnt an eine Zeile in Tim Bendzkos Song \emph{Wenn Worte meine Sprache wären}.}
  \item Was dieses Bild ausdrückt, könnten wir durch sprachliche Äußerungen alleine nicht einfangen.\footnote{So war beispielsweise der abstrakte Maler Mark Rothko der Auffassung, dass seine Bilder \enquote{grundlegendste menschliche Emotionen ausdrücken -- Tragik, Verzückung, Verhängnis und so weiter} und dabei über die Möglichkeiten bloßer Sprache hinausgehen. Mark Rothko zitiert nach\cite{hodge2019}, S. 146.}
  \item Es ist unmöglich, einem Blinden zu sagen, wie es ist, Farben zu sehen.
  \item Erfahrung aus erster Hand hat manche Aspekte, die bei der Übersetzung in sprachliche Äußerungen immer verloren gehen.\autocite[Vgl.][]{ranalli2020}
\end{itemize} % TODO: Reale Beispiele: Freges Beispiel; Vielleicht noch was aus Religion?

Diese Sätze scheinen zu behaupten, dass da etwas jenseits einer Grenze von Sprache liegt. Wie sollen wir uns so eine Grenze vorstellen? Was liegt innerhalb und was außerhalb von ihr? Und wie können wird diese Grenze verstehen, wenn wir mit Sprache über sie reden?

Ludwig Wittgenstein versucht mit seinem \emph{Tractatus logico-philosophicus} (im Folgenden TLP), all diese Fragen zu beantworten. Denn er braucht eine Grenze von Sprache für sein übergeordnetes Projekt: Die Probleme der Philosophie im Wesentlichen endgültig zu lösen. In dieser Arbeit werde ich Wittgensteins TLP als Versuch, eine Grenze von Sprache zu ziehen, rekonstruieren und evaluieren. Dazu werde ich drei Interpretationen des TLP diskutieren: Peter Hackers substanzielle Lesart, James Conants resolute Lesart und Daniele Moyal-Sharrocks Versuch, zwischen substanziellen und resoluten Lesarten zu vermitteln.

Meine Arbeit wird mit einer historischen Herleitung beginnen. Denn, um Wittgensteins Philosophie und ihre Stärken und Probleme zu verstehen, ist es wichtig, sie mit denen von Gottlob Frege und Bertrand Russell zu vergleichen. Danach werde ich Wittgensteins Gedanken zuerst naiv rekonstruieren. Diese naive Rekonstruktion scheitert aber an den \emph{Leiter-Problemen}, die sich aus Wittgensteins Philosophie ergeben. Daher werde ich mich den Ansätzen von Hacker, Conant und Moyal-Sharrock zuwenden. Leider können auch sie keine Interpretation anbieten, die die Probleme zufriedenstellend löst. Ich werde also zeigen, dass wir mit Wittgenstein nicht verstehen können, wo die Grenzen unserer Sprache liegen.


% TODO: unaussprechbar / unbeschreibbar / unbeschreiblich

% TODO: Was ist mit unkonventioneller Sprache: Metaphern, Analogien, Ironie, (Sprechakte wären Anachronistisch... andererseits schreibe ich eine systematische Arbeit, keine historische)

% Grundgedanke: Erst mystische Erfahrungen skizzieren, dann dafür argumentieren, dass sie kulturell konstruiert und gar nicht soo mystisch sind. (Erst Augustinus, dann neueres Zeug)

% Eine allgemeine Theorie sprachlicher Bedeutung muss diese Beispiele einordnen. Sie muss erklären, warum es auf den ersten Blick so scheint, als machten wir Erfahrungen jenseits der Grenzen von Sprache. Sie muss außerdem erklären, ob diese erste Intuition korrekt ist, oder ob wir sie überdenken müssen und warum.

% TODO: Vielleicht: \subsection{Unwahrnehmbares}

\section{Historische Einordnung}

Wittgensteins TLP lässt sich kaum losgelöst von seinem historischen Kontext verstehen. So schreibt er selbst, dass den TLP \enquote{[vielleicht nur derjenige verstehen wird], der die Gedanken, die darin ausgedrückt sind -- oder doch ähnliche Gedanken -- schon selbst einmal gedacht hat.} Insbesondere schuldet Wittgenstein \enquote{den großartigen Werken Freges und den Arbeiten [seines] Freundes Herrn Bertrand Russell einen großen Teil der Anregung zu [seinen] Gedanken}.\footnote{Vgl. \cite{wittgenstein1922}, Vorwort.} Im Folgenden werde ich also grob umreißen, welche Fragen sich Frege und Russell stellten und wie sie versuchten, diese zu beantworten.

\subsection{Frege}

Beginnen wir mit Frege. Freges Lebenswerk besteht in dem Versuch einer Reduktion der Mathematik auf die Logik.\footnote{Vgl. \cite{zalta2020}, Einleitung. Das Projekt der Reduktion der Mathematik auf die Logik wird heute als \enquote{Logizismus} bezeichnet.} Motiviert ist Frege zu diesem Projekt in großen Teilen durch seinen Dissens mit Immanuel Kant über das Wesen mathematischen Wissens. Kant argumentiert in seiner \emph{Kritik der reinen Vernunft} für zwei Dimensionen, in denen sich unser Wissen unterscheidet. Einerseits ist es entweder a priori oder a posteriori, also entweder ohne, oder nur durch Erfahrung zugänglich. Andererseits ist es entweder analytisch oder synthetisch. Es ist analytisch, wenn das, was durch ein Prädikat ausgesagt wird, bereits im Begriff des Subjekts steckt. Ansonsten ist es synthetisch. Der Satz \enquote{Alle Junggesellen sind unverheiratet.} ist beispielsweise ein analytischer Satz, weil in \enquote{Jungesellen} bereits steckt, dass sie unverheiratet sind.\autocite[Vgl.][]{kant1781} Anders ausgedrückt: Analytisches Wissen ist rein sprachliches Wissen. Es sind Sätze, die wir alleine aufgrund der Begriffe, in denen wir sie formulieren, als wahr erkennen. Kant ordnet mathematisches Wissen als synthetisch a priori ein. Ihm zufolge müssen wir uns mathematische Phänomene konkret und anschaulich vorstellen und können erst an diesen Vorstellungen lernen, was mathematisch wahr ist. Entgegen dessen will Frege zeigen, dass Mathematik analytisch a priori ist. Nach ihm handelt es sich bei Mathematik also um rein sprachliche Wahrheiten. Um das zu belegen, will Frege ausgehend von einigen wenigen grundlegendsten \emph{logischen} Denkgesetzen in vielen winzigen Schritten die Arithmetik herleiten. So will er zeigen, dass wir bei keinem dieser Schritte anschaulich denken müssen.\autocite[Vgl.][IV]{frege1879}

Doch Frege fällt auf, dass die Alltagssprache \emph{für diesen Zweck} nicht präzise genug ist. Wenn wir informell sprechen, machen wir zum Beispiel oft Gedankensprünge, die bei Freges Reduktion der Mathematik den Eindruck erwecken könnten, dass dieser unbemerkt doch etwas Anschauliches einfließen lässt. Außerdem haben Begriffe in der Alltagssprache Färbungen. Ich kann einen Hund zum Beispiel auch \enquote{Köter} nennen und damit etwas anderes ausdrücken als mit \enquote{Hund}, auch wenn das für den reinen Inhalt und die logische Schlussfolgerung keinen Unterschied macht. Frege erfindet also eine formale Notation, die er \enquote{Begriffsschrift} nennt, um Schlussfolgerungen mit größtmöglicher Präzision darzustellen:

\begin{quote}
  Indem ich diese Forderung auf das strengste zu erfüllen trachtete, fand ich ein Hindernis in der Unzulänglichkeit der Sprache, die bei aller entstehenden Schwerfälligkeit des Ausdrucks doch, je verwickelter die Beziehungen wurden, desto weniger die Genauigkeit erreichen liess, welche mein Zweck verlangte. Aus diesem Bedürfnisse ging der Gedanke der vorliegenden Begriffsschrift hervor. Sie soll also zunächst dazu dienen, die Bündigkeit einer Schlusskette auf die sicherste Weise zu prüfen und jede Voraussetzung, die sich unbemerkt einschleichen will, anzuzeigen, damit letztere auf ihren Ursprung untersucht werden könne. Deshalb ist auf den Ausdruck alles dessen verzichtet worden, was für die \emph{Schlusskette} ohne Bedeutung ist.\autocite[][IV]{frege1879}
\end{quote}

Diese Begriffsschrift ist eine revolutionäre neue Logik. Ausgehend von ihr zeigt Frege dann, wie sich weite Teile der Arithmetik herleiten lassen.\footnote{Das sind nicht Freges einzige Innovationen. Besonders nennenswert ist auch noch seine Definition mathematischen Beweisens. Vgl. \cite{zalta2020}, Einleitung.} An dieser Stelle ist es wichtig, anzumerken, dass Frege die Begriffsschrift als Werkzeug mit einem besonderen Zweck versteht. Er vergleicht sie in der Hinsicht mit einem Mikroskop. Sie ist der Alltagssprache (wie das Mikroskop dem Auge) für die meisten Aufgaben unterlegen. Wenn es aber auf extreme Präzision ankommt, glänzt sie.\autocite[Vgl.][V]{frege1879} Die Begriffsschrift soll besonders gut dazu geeignet sein, Schlussfolgerungen explizit und präzise darzustellen. Und das soll in erster Linie der Erforschung der Grundlagen der Mathematik dienen. Frege erkennt aber auch den möglichen Nutzen für Naturwissenschaft und Philosophie. Zur Philosophie schreibt er:

\begin{quote}
  Wenn es eine Aufgabe der Philosophie ist, die Herrschaft des Worts über den menschlichen Geist zu brechen, indem sie die Täuschungen aufdeckt, die durch den Sprachgebrauch über die Beziehungen der Begriffe oft fast unvermeidlich entstehen, indem sie den Gedanken von demjenigen befreit, womit ihn allein die Beschaffenheit des sprachlichen Ausdrucksmittels behaftet, so wird meine Begriffsschrift, für diese Zwecke weiter ausgebildet, den Philosophen ein brauchbares Werkzeug werden können. Freilich giebt auch sie, wie es bei einem äussern Darstellungsmittel wohl nicht anders möglich ist, den Gedanken nicht rein wieder; aber einerseits kann man diese Abweichungen auf das Unvermeidliche und Unschädliche beschränken, andererseits ist schon dadurch, dass sie ganz andrer Art sind als die der Sprache eigenthümlichen, ein Schutz gegen eine einseitige Beeinflussung durch eines dieser Ausdrucksmittel gegeben.\autocite[][VI-VII]{frege1879}
\end{quote}

Bei dieser Textstelle will ich besonders betonen, wie Frege die Begriffsschrift relativiert. Sie gibt \enquote{den Gedanken nicht rein wieder} und müsste zuerst für die jeweiligen \enquote{Zwecke weiter ausgebildet} werden. Hier antizipiert Frege, dass wir für verschiedene Aspekte unserer natürlichen Sprache, verschiedene Logiken konstruiert müssen, die diese modellieren.\footnote{Ich meine hier zum Beispiel Modallogiken, die wir heute verwenden, um unsere Rede von Möglichkeit und Notwendigkeit einzufangen; Deontische Logiken, die unseren normativen Diskurs formalisieren oder auch Mehrwertige- bzw. Fuzzylogiken, die vage Begriffe und Sätze mit Wahrheitswertlücken formalisieren.} Keine von diesen Logiken gibt den Gedanken \enquote{rein} wieder. Doch die Begriffsschrift hilft uns schon allein deshalb beim Philosophieren, weil sie unsere Gedanken auf andere Weise verfälscht als die Alltagssprache.\footnote{Eine zu ausführliche historische Einordnung läuft Gefahr, erst bei Platon oder den Vorsokratikern halt zu machen und dabei den systematischen Punkt zu verpassen. Daher habe ich meine hier bei Frege abgebrochen. Dennoch sei erwähnt, dass sich Frege mit dem Gedanken, dass wir philosophische Probleme durch eine Übersetzung der Positionen in eine formale Notation lösen können, auf den Rationalisten des 17. Jahrhunderts Gottfried Wilhelm Leibniz beruft. Vgl. z.B. \cite{leibniz1677}.}

Damit ist sich Frege schon von Anfang an bewusst, dass seine Begriffsschrift Grenzen hat. Bei seinem Versuch, die Begriffsschrift zu erläutern, stößt er dabei umso heftiger an diese. Denn es stellt sich heraus, dass sich die Begriffsschrift nicht mit den Mitteln der Begriffsschrift erläutern lässt. Das zeigt sich besonders klar an Freges Debatte mit Benno Kerry über Begriff und Gegenstand.\autocite[Vgl. im Folgenden][]{frege1892}

Während Frege seine Begriffsschrift entwickelt, stellt er fest, dass ein Satz, um sinnvoll zu sein, einen gesättigten und einen ungesättigten Teil benötigt. Ein Teil eines Satzes muss unabgeschlossen sein, er muss so etwas wie eine Lücke haben, damit ein anderer in ihn passen kann. Frege veranschaulicht das am Beispiel des Satzes \enquote{Die Zahl 2 ist eine Primzahl.} Diesen Satz können wir so auffassen, dass mit ihm der Gegenstand auf den \enquote{Die Zahl 2} verweist, unter den Begriff gefasst wird, den wir durch \enquote{... ist eine Primzahl} ausdrücken. Die Zahl 2 ist dabei einfach nur die Zahl 2. Sie ist ein in sich abgeschlossenes Ding. \emph{... ist eine Primzahl} dagegen ist nicht in dieser Weise abgeschlossen, denn als Begriff fehlt ihm noch etwas. Er hat eine Lücke, in die sich die Zahl 2 fügt. Frege ist der Auffassung, dass sich Gegenstand und Begriff in dieser Weise grundlegend unterscheiden und dass es unmöglich ist, dass etwas sowohl Gegenstand als auch Begriff ist.\autocite[Vgl.][13]{frege1892}

An dieser Stelle widerspricht Kerry, indem er dafür argumentiert, dass der Begriff Primzahl nur ein weiterer Gegenstand ist, der unter weitere Begriffe fällt. Wir können zum Beispiel Sätze formulieren wie: \enquote{Der Begriff Primzahl ist schwer zu erlenen.} In diesem Satz ist der Begriff \emph{Primzahl} ein Gegenstand, der unter den Begriff \emph{... ist schwer zu erlenen} gefasst wird. Dabei bleibt er aber ein Begriff.\autocite[Vgl.][3-5]{frege1892}

Frege wirft Kerry angesichts dieser Kritik vor, seine Verwendung des Worts \enquote{Begriff} nicht zu verstehen. Kerrys Einwand sei nur mit einer psychologischen Verwendung von \enquote{Begriff} verständlich. Nach dieser Verwendung ist ein Begriff ein mentaler Zustand, etwas, das wir uns merken und wieder vergessen können. Frege gehe es aber um eine logische Verwendung von \enquote{Begriff}. Hier geht es ausschließlich um Teile dessen, was ein Satz ausdrückt. Einen Begriff in diesem Sinn kann man also gar nicht von dem Satz trennen, in dem er vorkommt.\autocite[Vgl.][2]{frege1892}

Kommen wir, um dieses logische Verständnis von Begriffen besser zu verstehen, noch einmal zurück zu Freges Beispiel \enquote{Die Zahl 2 ist eine Primzahl}. Würden wir \emph{... ist eine Primzahl} auch einfach nur als Namen für einen Begriffs-Gegenstand behandeln, hätten wir, wenn wir die beiden Namen aneinanderreihen, einen Satz wie: \enquote{Die Zahl 2 der Begriff Primzahl.} Diesem Satz lässt sich kein Sinn entnehmen. Und das Ganze lässt sich nicht einmal korrigieren, indem wir noch die Beziehung des Fallens unter einen Gegenstand einführen. Denn entweder, wir nehmen mit \enquote{Die Zahl 2 fällt unter den den Begriff Primzahl.} an, dass \emph{... fällt unter ...} ungesättigt ist, oder wir formulieren einen Satz wie: \enquote{Die Zahl 2 die Beziehung des Fallens unter einen Gegenstand der Begriff Primzahl.}, dem sich wieder kein Sinn entnehmen lässt.\autocite[Vgl.][13]{frege1892}

Damit ist Freges Ergebnis, dass ein sinnvoller Satz einen gesättigten Gegenstand und einen ungesättigten Begriff benötigt. Dieses Ergebnis hat aber eine rätselhafte Konsequenz: Ein Satz wie \enquote{Der Begriff \emph{Primzahl} ist ungesättigt.}, kann nicht auf den Begriff verweisen, der in einem Satz wie \enquote{Die Zahl 2 ist eine Primzahl.} vorkommt. Denn in \enquote{Der Begriff \emph{Primzahl} ist ungesättigt.} funktioniert \enquote{Der Begriff \emph{Primzahl}} eben nicht als ungesättigter Teil, dessen Lücke noch durch eine Zahl gefüllt werden muss. Frege erklärt uns also, wie eine bestimmte Sprache funktioniert. Doch diese Erklärung kann nicht erklären, wie sie selbst verstanden werden kann. Doch Frege sieht darin kein großes Problem. Es brauche nur etwas Wohlwollen der LeserIn und ein Körnchen Salz:

\begin{quote}
  Der Verständigung mit dem Leser steht freilich ein eigenartiges Hinderniss im Wege, dass nämlich mit einer gewissensprachlichen Nothwendigkeit mein Ausdruck zuweilen, ganz wörtlich genommen, den Gedanken verfehlt, indem ein Gegenstand genannt wird, wo ein Begriff gemeint ist. Ich bin mir völlig bewusst, in solchen Fällen auf ein wohlwollendes Entgegenkommen des Lesers angewiesen zu sein, welcher mit einem Körnchen Salz nicht spart.\autocite[][12-13]{frege1892}
\end{quote}

Was die Grenzen unserer Sprache angeht, können wir für Frege also Folgendes festhalten. Frege interessiert sich für Fragen der Philosophie der Mathematik. Er glaubt, dass unsere natürliche Alltagssprache noch nicht optimal geeignet ist, um diese Fragen zu erforschen. Daher ersetzt er unsere natürliche Sprache durch eine formale, von ihm konstruierte Idealsprache: die Begriffsschrift. Diese dient ihm auch dazu, zu verstehen, was in unserer natürlichen Sprache vorgeht. Erstens beansprucht diese Begriffsschrift aber nicht, alles einzufangen, was in unserer Sprache vorgeht. Frege sieht sie nur als ein Mittel, um kleinteilige, technische Argumente explizit und scharf darzustellen. Und zweitens ist sich Frege bewusst, dass die Begriffsschrift nicht in der Lage ist, vollständig über sich selbst zu sprechen. Dazu brauchen wir unsere Alltagssprache, etwas Wohlwollen bei der Interpretation und eine Prise Salz. Frege interessiert sich also kaum für die Frage, die hier zur Debatte steht. Er fragt nicht nach den Grenzen von Sprache schlechthin, sondern konstruiert eine formale Sprache, weil ihm die Alltagssprache für manche Zwecke zu unpräzise ist. Dabei sind beide Sprachen auf unterschiedliche Weise begrenzt.

\subsection{Russell}

Wittgensteins zweiter großer historischer Vorläufer, Bertrand Russel, schließt direkt an Freges \emph{Grundgesetze der Arithmetik} an.\autocite[Vgl. im Folgenden][insbesondere Abschnitt 2: Russell’s Work in Logic]{irvine2020} Denn Russell entdeckt, dass eins der Axiome, auf dem Freges \emph{Grundgesetze der Arithmetik} basieren, inkonsistent ist.\footnote{Neure Forschung zeigt, dass sich ein beträchtlicher Teil von Freges Errungenschaften retten lässt, wenn man nur seine Herleitung für das Grundgesetz V außer Acht lässt, und dieses stattdessen als Axiom behandelt. Vgl. für einen Überblick \cite{zalta2020a}.} \enquote{Russells Paradoxie}, wie das Problem heute genannt wird, entsteht, weil Freges System es erlaubt, die Menge all der Mengen zu bilden, die nicht Element von sich selbst sind. Nennen wir diese Menge $R$. Nun gibt es nur zwei Möglichkeiten: Entweder $R$ ist Element von $R$, oder eben nicht. Wenn $R$ Element von $R$ ist, entsteht ein Widerspruch, weil $R$ dann ein Element enthält, das Element von sich selbst ist. Ist $R$ dagegen kein Element von $R$ entsteht ein Widerspruch, weil $R$ dann ein Element fehlt, das nicht Element von sich selbst ist.

Russell versucht dieses Problem durch die \emph{Verzweigte Typentheorie}\footnote{Für Russells Theorie der Typen vgl. \cite{russell1903}; \cite{russell1903} und \cite{alfrednorthwhiteheadbertrandrussell1910}, Introduction: Chapter II: The Theory of Logical Types. Zitiert nach \cite{irvine2020}. Eine eine leichter verständliche Darstellung der Typentheorie in zeitgenössischer Notation mithilfe von Alonzo Churchs Formalismus findet sich außerdem in \cite{linsky2019}, 4.1.2 The \enquote{Ramified} Theory of Types.}\footnote{Russell und Whitehead unterscheiden selbst noch nicht explizit zwischen der einfachen- und der verzweigten Typentheorie. Der Name, den ich hier verwende, geht auf \cite{chwistek1921} und \cite{ramsey1931} zurück. Zitiert nach \cite{linsky2019}.} zu lösen, die er gemeinsam mit Alfred North Whitehead in der \emph{Principia Mathematica} formuliert. In dem monumentalen Werk führen Russell und Whitehead Freges Programm weiter fort: Sie versuchen, die Mathematik auf die Logik zu reduzieren. Heute ist unumstritten, dass dieser Versuch zumindest widerspruchsfrei ist. Russell und Whitehead führen für ihre Logik aber mit dem \emph{axiom of infinity} und dem \emph{axiom of reducibility} zwei Axiome ein, aufgrund von denen es heute kontrovers bleibt, ob sie die Mathematik wirklich auf Logik alleine reduzieren konnten, oder ob sie doch weitere Annahmen voraussetzen.\autocite[Vgl.][Einleitung und 1. Overview]{linsky2019}

Frege erlaubt es noch, Prädikate mit beliebigen Begriffsumfängen zu bilden, weshalb sich für ihn Russells Paradoxie konstruieren lässt. Russell und Whitehead schränken mit der Typentheorie ihre formale Sprache auf bestimmte Arten ein, um diese und andere Paradoxien zu vermeiden. Verschiedene Ausdrücke haben demnach je einen logischen Typ. Dabei legt die Typentheorie fest, dass ein Prädikat nur auf etwas bezogen werden kann, das einen anderen logischen Typ hat als es selbst. Russell und Whitehead modellieren Prädikate hier durch propositionale Funktionen, also als Funktionen, die erst zu Propositionen werden, wenn sie durch bestimmte Argumente ergänzt werden. In diesem Geiste könnte man zum Beispiel das Prädikat \enquote{ist sterblich} durch eine Funktion interpretieren, deren einziges Argument vom Typ $\iota$, dem Typen für Individuen sein muss. Die Funktion hätte dann den Typ $(\iota)/1$. Sie ist also nicht auf sich selbst anwendbar.\footnote{Russell und Whitehead entwickeln selbst keine Notation für Typen, weil es ihnen mit der Reduktion der Mathematik auf die Logik um eine Frage geht, die so abstrakt ist, dass es nicht nötig ist, auf die Typen einzelner sprachlicher Äußerungen Bezug zu nehmen. So eine Notation lohnt sich aber, um die Gedanken anschaulich zu erklären. Daher verwende ich hier inspiriert von Bernard Linsky und Andrew David Irvine die Typen-Notation Alonzo Churchs. Vgl. \cite{linsky2019}, 4.1.2 The \enquote{Ramified} Theory of Types, Church’s (1976) formulation of the logic of PM with \emph{r-types} und \cite{linsky2021}, 5. The Missing Notation for Types and Orders.} Wenn wir jetzt versuchen, Russells Menge mithilfe einer solchen Funktion zu bilden, geraten wir in Schwierigkeiten. Denn hierfür müssten wir eine Funktion bilden, die sich selbst und Funktionen ihres eigenen Typs als Argument haben kann, was die Typentheorie verbietet.

Wie bereits bei Frege stoßen wir auch bei der formalen Sprache der Prinzipia Mathematica auf Ausdrucksschwierigkeiten. Denn die Typentheorie lässt sich nur erklären, wenn wir über Dinge beliebiger Typen quantifizieren können. Genau das verbietet sie aber. Denn jedes Argument der Funktionen, aus denen unsere Sätze bestehen, muss ja einen bestimmten, vorher schon festgelegten Typ haben. Auch die Sprache der Prinzipia Mathematica ist also nicht in der Lage, adäquat über sich selbst zu reden.

So weit also zu Russells Logik und seiner Philosophie der Mathematik. Kommen wir jetzt zu einem verwandten Bereich: der Sprachphilosophie. Auch hier ähnelt Russells Herangehensweise der Freges. Er bemerkt philosophische Probleme, die vorerst in gewöhnlicher Alltagssprache formuliert sind. Indem er die Sätze unserer Alltagssprache analysiert, schafft er es aber, die Probleme aufzulösen. Betrachtet man die analysierten Sätze, stellen sich die philosophischen Fragen gar nicht mehr. Frege hatte in dieser Manier seinen Streit mit Kant über die Analytizität der Mathematik ausgefochten. Er betrachtete auch andere einzelne philosophische Rätsel, wie zum Beispiel die Frage danach, was eine Anzahl von Dingen ist.\autocite[Vgl.][§35-53 und §62-67]{frege1884} Für Russell werde ich seine Lösung des Rätsels des Nicht-Seins betrachten\autocite[Vgl. im Folgenden][]{russell1905}:

\begin{quote}
  \textbf{Rätsel des Nicht-Seins:} Wovon reden wir, wenn wir bestreiten, dass es etwas gibt? Wenn ich so etwas sage wie: \enquote{Die runde eckige Kuppel gibt es nicht.}, dann muss dieser Satz schließlich von irgendetwas handeln. Von der runden eckigen Kuppel kann er aber nicht handeln, da ja gerade bestritten wird, dass es diese überhaupt gibt!\footnote{Russell diskutiert dieses Rätsel in einer etwas komplizierteren Formulierung, bei der nicht direkt die Existenz eines Gegenstands sondern die Existenz von Unterschieden zwischen zwei Gegenständen bestritten wird. Ansonsten handelt es sich aber um dasselbe Rätsel. Vgl. \cite{russell1905}, S. 485, Absatz 3 zur ursprünglichen Formulierung und S. 490-491 Lösung des Rätsels. Meine etwas zugänglichere Formulierung hier geht auf \cite{quine1948} zurück.}
\end{quote}

Um dieses (und andere) Rätsel zu lösen, entwickelt Russell eine eigene Theorie der Kennzeichnung. Unter einer Kennzeichnung versteht er Phrasen wie \enquote{ein Mann}, \enquote{jeder Mann}, \enquote{die gegenwärtige Königin von England}, \enquote{der gegenwärtige König von Frankreich} oder eben \enquote{die runde eckige Kuppel}. Und seine Theorie soll uns verraten, wie Sätze zu verstehen sind, in denen solche Kennzeichnungen vorkommen. Betrachten wir die Analyse des rätselhaften \textbf{Kuppel}-Satzes, die sich aus Russells Theorie ergibt:

\claim{Kuppel, Alltagssprache} Die runde eckige Kuppel gibt es nicht.

\claim{Kuppel, Russells Analyse} Es ist nicht der Fall, dass es ein $x$ gibt, so dass $x$ sowohl rund als auch eckig und dazu eine Kuppel ist und dass $x$ einzigartig ist.\footnote{Diese Formulierung ist angelehnt an: \enquote{We can now see also how to deny that there is such an object as the difference between A and B in the case when A and B do not differ. If A and B do differ, there is one and only one entity $x$ such that \enquote{$x$ is the difference between A and B} is a true proposition.}, \cite{russell1905}, S. 490.}

\enquote{$x$} wird dabei als \enquote{vollkommen unbestimmte Variable} verstanden. Das heißt, es kann für beliebiges stehen. Wenn es auch nur irgendetwas gibt, dass als einziges Ding rund eckig und eine Kuppel ist -- sei das eine Katze, ein Schuh oder eine Person -- dann ist der Satz falsch. Natürlich gibt es so etwas aber nicht. Der Satz ist also wahr.

Der Trick bei dieser Analyse ist, dass die Phrase \enquote{die runde eckige Kuppel} sich nach Russell für sich alleine noch auf nichts bezieht. Sie erlangt erst im Kontext des Satzes Bedeutung. In diesem Satz verschwindet sie aber mit der Analyse. Indem wir die Variable verwenden, sagen wir letztendlich nur, dass wir alle Gegenstände durchgehen können und dabei keinen finden werden, der (als einziger) sowohl rund als auch eckig als auch eine Kuppel ist. Und das geht, ohne dass wir uns je auf eine runde eckige Kuppel beziehen.

Damit löst Russell also ein philosophisches Problem durch logische Analyse. Sätze aus dem Alltag sind rätselhaft. Doch sobald wir sie in analysierter Form betrachten, stellen sich unsere Fragen nicht mehr.

\subsection{Zusammenfassung}

Trotz ihrer Differenzen teilen Frege und Russell damit wichtige Annahmen und Vorhaben. Beide interessieren sich dafür, die Mathematik philosophisch zu reflektieren und versuchen zu zeigen, dass diese sich auf die Logik reduzieren lässt. Sie reflektieren und verfeinern dabei die logischen Formalismen ihrer Zeit und nutzen diese, um philosophische Rätsel zu lösen. Wie diese Lösungen funktionieren, haben wir am Beispiel von Russells Lösung des \textbf{Rätsels des Nicht-Seins} gesehen. Die Sätze, die für das Rätsel sorgen, werden mithilfe von Russells Theorie in eine präzise Idealsprache übersetzt. Sobald wir sie in dieser Idealsprache sehen, stellt sich das Rätsel dann nicht mehr. % TODO: Muss nochmal "On Denoting" lesen, vielleicht schreibt Russell gar nicht über dieses Rätsel

Dazu kommt, dass beide mit vergleichbaren Ausdrucksschwierigkeiten zu kämpfen haben. Die Idealsprachen, die sie entwickeln, können sich nicht selbst erklären, sondern brauchen unsere Alltagssprache und ein wohlwollendes Körnchen Salz, um verstanden zu werden. Das ist aber auch nicht schlimm, weil diese Idealsprachen selbstreflektiert für einen bestimmten Rahmen geschaffen werden. Sie sind wie Mikroskope. Sie eigenen sich, um in einem festgesteckten Bereich zu forschen, wir sollten sie aber nicht anstatt unserer Augen verwenden und wir brauchen Augen, um sie zu bauen und um sie zu gebrauchen.\autocite[Vgl. zu diesem Vergleich][V]{frege1879}

Wenn Wittgenstein schreibt, dass sein Buch \enquote{[vielleicht nur derjenige verstehen wird], der die Gedanken, die darin ausgedrückt sind -- oder doch ähnliche Gedanken -- schon selbst einmal gedacht hat}, dann meint er also, dass sein Buch nur der verstehen wird, der schon über die Grundlagen der Mathematik und Logik nachgedacht hat. Insbesondere wird sein Buch aber nur der verstehen, für den diese Reflexionen über das Wesen der Logik ein Ansatz waren, philosophische Probleme zu lösen. Es wird vorausgesetzt, dass der Leser Formalismen wie die Russells und Freges in der Hoffnung konstruiert hat, dass manche philosophische Probleme sich dadurch lösen lassen, dass problematische Sätze in derart ideale Sprachen übersetzt werden.

\section{Rekonstruktion des TLP}

\subsection{Wittgensteins Mission}

Der TLP ist ein etwas eigenwilliges philosophisches Werk. Frege und Russell präsentieren ihren LeserInnen sorgfältig geordnete Kapitel und Unterkapitel, in denen sie ihre Thesen argumentativ und klar ausarbeiten. Neue Formalismen oder Fachbegriffe werden oft erklärt und definiert. Die LeserIn wird an der Hand genommen und Station für Station durch den Gedankengang geführt. Wittgensteins TLP besteht dagegen aus nummerierten Sätzen mit Bemerkungen und Bemerkungen zu Bemerkungen. Formalismen und Begriffe werden oft einfach vorausgesetzt, was die Thesen und die Argumente sind, ist häufig nicht offensichtlich. Angesichts dessen ist das Vorwort des TLP erfrischend klar. Hier verrät Wittgenstein uns in knappen Worten, was für ein Projekt er mit dem TLP verfolgt. Ich werde also im Folgenden herausarbeiten, was Sinn und Zweck des TLP sind, indem ich das Vorwort detailliert interpretiere.\footnote{Vgl. im Folgenden \cite{wittgenstein1922}, Vorwort.}

Nachdem Wittgenstein uns warnt, wie voraussetzungsreich der TLP ist, schreibt er, dass sein Buch \enquote{die philosophischen Probleme} behandelt. Seine motivierende Fragestellung ist also:

\claim{Motivierende Fragestellung TLP} Wie lassen sich die philosophischen Probleme lösen?

Wittgenstein ist nicht der erste Philosoph, der versucht, diese Frage zu beantworten. Er verfolgt aber einen besonderen Ansatz. Philosophische Probleme basieren auf logischem Irrtum:

\begin{quote}
  Das Buch behandelt die philosophischen Probleme und zeigt -- wie ich glaube --, daß die Fragestellung dieser Probleme auf dem Mißverständnis der Logik unserer Sprache beruht.\footnote{\cite{wittgenstein1922}, Vorwort.}
\end{quote}

In dieser Hinsicht ist Wittgenstein inspiriert von Frege und Russell. Alle drei glauben, dass sich viele philosophische Probleme durch eine sorgfältige logische Analyse auflösen. Während Frege und Russell aber noch einzelne Probleme behandeln, will der junge Wittgenstein die ganze Philosophie auf einmal lösen. Er will aufzeigen, dass philosophische Probleme, gerade weil und insofern sie \emph{philosophische} Probleme sind, in logischem Irrtum bestehen. Schließlich schreibt er, dass die Fragestellung \enquote{dieser}, also genau der und aller philosophischer Probleme gemeint ist.

Weiter schreibt Wittgenstein, dass er die philosophischen Probleme lösen will, indem sein Buch dem Ausdruck von Gedanken eine Grenze zieht. Dem Denken könne man dagegen keine Grenze ziehen, da man dazu beide Seiten der Grenze denken können müsse. Er schreibt: \enquote{Die Grenze wird also nur in der Sprache gezogen werden können und was jenseits der Grenze liegt, wird einfach Unsinn sein.} Diese Ausführung verrät uns mehr über die Lösung der philosophischen Probleme, die er anbietet. Diese Lösung ist eine Grenze sprachlichen Ausdrucks. Die Fragestellung philosophischer Probleme liegt jenseits dieser Grenze. Sprachliche Versuche, solche Fragestellungen (geschweige denn ihre Antworten) zu artikulieren sind also \enquote{einfach Unsinn}. Sie sind scheiternde Versuche, etwas zu sagen, was nicht sagbar ist. Damit beantwortet Wittgenstein seine motivierende Fragestellung, indem er die Folgende untersucht:

\claim{Fragestellung TLP} Unter welchen Bedingungen kann ein Gedanke ausgedrückt werden?

Letztendlich weist Wittgenstein philosophische Fragestellungen und Antworten also mit dem Nachweis ihrer Unsinnigkeit zurück. Nachdem er den TLP fertig stellte, ließ er die Philosophie ruhen und wurde Grundschullehrer. Wenn er schreibt, dass sein Nachweis, dass Philosophie qua Philosophie unsinnig ist, die Probleme der Philosophie löst, dann müssen wir diese Lösung also zumindest so verstehen, dass sie für ihn Grund genug war, sich nicht weiter mit Philosophie zu beschäftigen. Das heißt, dass für Wittgenstein Unsinnigkeit ein Grund ist, eine Fragestellung und ihre Antworten zurückzuweisen. Was unsinnig ist, weil es jenseits der Grenzen der Sprache liegt, über das sollen wir keine Bücher lesen oder schreiben, wir sollen es nicht diskutieren, es lohnt sich nicht, darüber zu reden. Nach dieser Interpretation Wittgensteins wird bei ihm ein Ansatz also, insofern er ein Beitrag zu einer Debatte ist, durch den Nachweis seiner Unsinnigkeit \emph{disqualifiziert}.

Zum Schluss des Vorworts sagt uns Wittgenstein schließlich, worin er den Wert des TLP sieht. Erstens ist der TLP wertvoll, weil in ihm Gedanken klar ausgedrückt werden (Auch, wenn Wittgenstein \enquote{weit hinter dem Möglichen zurückgeblieben} ist).\footnote{Vgl. \cite{wittgenstein1922}, Vorwort.} Dieser Wert ist epistemisch: Je klarer eine Position artikuliert ist, desto leichter können wir feststellen, ob sie wahr ist, oder nicht. Das hilft uns wiederum, ähnliche Positionen zu erkennen und einzuordnen. Eine klar ausgedrückte Position bringt uns also, unabhängig davon, ob sie wahr ist oder nicht, der Wahrheit näher.

Zweitens scheint Wittgenstein \enquote{die Wahrheit der hier mitgeteilten Gedanken unantastbar und definitiv}. Durch sie werden \enquote{die Probleme im Wesentlichen endgültig gelöst}. Doch der Wert dessen ist, dass das \enquote{zeigt, wie wenig damit getan ist, daß diese Probleme gelöst sind}. Hier verrät uns Wittgenstein also, dass er glaubt, dass sein Projekt gelingt. Er glaubt, dass er es schafft, eine Grenze sprachlichen Ausdrucks zu ziehen und dass sich aus dieser ergibt, dass philosophische Diskussionen unsinnig und damit abzulehnen sind. Doch damit ist noch sehr wenig getan. Wieso, werden wir erst verstehen können, wenn ich Wittgensteins Lösung rekonstruiert und auf ein paar philosophische Probleme bezogen habe. Für jetzt ist es aber wichtig festzuhalten, dass der TLP Wittgenstein zufolge wahre Gedanken ausdrückt.

Diese Interpretation mag auf den ersten Blick etwas radikal klingen. Sie könnte bei manchen LeserInnen den Anschein erwecken, dass ich hier einen Strohmann aus Wittgensteins Position mache, indem ich ihren Anspruch zu ambitioniert formuliere. Einem weniger radikalen Verständnis des Projekts zufolge, würde Wittgenstein zum Beispiel die Philosophie trotz ihrer Unsinnigkeit nicht delegitimieren. Vielleicht ist es ja in Ordnung, dass Philosophie unsinnig ist. Mit dieser Lesart wird aber nicht mehr klar, inwiefern der Nachweis ihrer Unsinnigkeit die philosophischen Probleme löst. Wir müssen hier schließlich Wittgensteins Handlungen erklären. Für ihn war der TLP Grund genug, um die Philosophie ruhen zu lassen. Ein Problem zu lösen, heißt in diesem Kontext also, es zurückzuweisen. Philosophische Fragen und Antworten zu formulieren basiert auf einem \enquote{Mißverständnis}, und Wittgenstein klärt diesen Irrtum auf.

Eine weitere Art, Wittgensteins Projekt abzuschwächen, wäre mit einem engeren Philosophiebegriff. Vielleicht meint Wittgenstein mit \enquote{Philosophie} nicht die reiche und vielfältige akademische Disziplin, wie sie heute an Philosophieinstituten praktiziert wird, sondern ein engeres Projekt, das nur die Arten von Fragen umfasst, mit denen sich Frege und Russell beschäftigten. Um auf diesen Einwand zu antworten, muss ich darauf eingehen, wie sich Lösungen philosophischer Probleme aus dem TLP ergeben. Und dazu muss ich zuerst den TLP rekonstruieren. Ich werde diesen Einwand also erst beantworten, nachdem ich im Folgenden erkläre, wie Wittgenstein im TLP eine Grenze sprachlichen Ausdrucks zieht.

% Weiterdenken: Was für eine Art von "sollen" in "worüber man nicht reden kann, darüber muss man schweigen"? Moralisch? Erkenntnistheoretisch? Grammatisch? (letzteres wäre trivial und nichtssagend; Moral unplausibel; muss Erkenntnistheoretisch sein und damit beißt sich der Löwe in den Schwanz)

\subsection{Wittgensteins \enquote{Bildtheorie} der Sprache} % TODO: Expliziter: Wittgenstein konstruiert eine formale Sprache. In deren Sätzen sollen sich logische Syntax, Semantik und Ontologie aller Sprachen zeigen

Kommen wir daher zu Wittgensteins Bildtheorie der Sprache. Diese wird zwar häufig Bild-\emph{Theorie} genannt, der Name ist aber eigentlich nicht treffend. Denn eine Theorie im herkömmlichen Sinn ist eine Menge von Aussagen, die bestimmte Phänomene erklären. Diese Aussagen könnten falsch sein, sind aber (im Optimalfall) wahrscheinlich wahr und werden manchmal sogar durch empirische Belege gestützt. So etwas ist Wittgensteins Bildtheorie nicht. Denn eine Theorie setzt immer schon eine Sprache voraus, in der sie formuliert ist, und Wittgenstein geht es gerade darum, solche Sprachen zu verstehen. Ganz in der Tradition von Frege und Russell erläutert Wittgenstein daher eine formale Sprache und keine Theorie. Bilder dienen dabei nur als Vergleich: Die formale Sprache, die Wittgenstein erläutert, bezieht sich so auf die Wirklichkeit, wie es Bilder tun.

Dabei orientiert sich Wittgenstein stark an seinen Vorläufern. Wie schon Frege und Russell will Wittgenstein Probleme unserer Alltagssprache beheben. Sie ist zu ambig, denn sie lässt unter anderem zu \enquote{daß dasselbe Wort auf verschiedene Art und Weise bezeichnet} und \enquote{daß zwei Wörter, die auf verschiedene Art und Weise bezeichnen, äußerlich in der gleichen Weise im Satze angewandt werden}. Daher folgert Wittgenstein: \enquote{Um diesen Irrtümern zu entgehen, müssen wir eine Zeichensprache verwenden, [...] die der logischen Grammatik - der logischen Syntax - gehorcht.}\footnote{Vgl. \cite{wittgenstein1922}, 3.323 und 3.325.} Auch Wittgenstein konstruiert also eine Idealsprache.

Er beginnt den TLP aber nicht mit der Erläuterung dieser Sprache, sondern indem er seine Ontologie schildert. Wie sich später zeigen wird, hängen diese eng zusammen. Beginnen wir also damit, was es Wittgenstein zufolge gibt, und wie das, was es gibt, strukturiert ist.

Wittgenstein zufolge besteht die Welt nicht aus Gegenständen sondern aus Tatsachen.\footnote{Vgl. \cite{wittgenstein1922}, 1.1-1.11.} Im Alltag beziehen wir uns mit \enquote{Gegenstand} häufig auf kleine feste Körper wie z.B. Laptops, Bücher oder Bücherregale. Tatsachen sind dagegen Beziehungen zwischen diesen Gegenständen. Zwei Beispiele wären, \emph{dass der Laptop angeschaltet ist}, oder \emph{dass die Bücher verstreut um das Bücherregal herum liegen}. Betrachtet man nur die Gegenstände alleine, ohne ihre Beziehungen in Betracht zu ziehen, kann man die Welt nicht begreifen. Mein Zimmer ist beispielsweise nicht nur eine Menge der Gegenstände darin. Es kommt auf die Beziehungen der Gegenstände zu einander an. Die Bücher liegen neben dem Regal, nicht darin. Damit gäbe es immer auch andere Weisen, Gegenstände miteinander in Beziehung zu setzen. Die Bücher könnten fein säuberlich geordnet im Regal stehen. Neben Gegenständen gibt es also noch verschiedene Verbindungen von Gegenständen, diese nennt Wittgenstein \enquote{Sachverhalte}. So ein Sachverhalt kann bestehen oder nicht bestehen und wenn er besteht, ist er auch eine Tatsache.\footnote{[Vgl. zu Gegenständen und Sachverhalten, \cite{wittgenstein1922}, 2-2.063.} Alle Tatsachen zusammen sind dann die Welt, alle Sachverhalte zusammen die Wirklichkeit.

An dieser Stelle bin ich vorerst von einem alltagssprachlichen Verständnis von \enquote{Gegenstand}, \enquote{Tatsache}, usw. ausgegangen. Wittgenstein lehnt sich mit seiner Verwendung dieser Begriffe aber nur entfernt an unsere Alltagssprache an. So würde er Bücher und Bücherregale nicht als \enquote{Gegenstände} bezeichnen und es wäre ihm zufolge auch keine Tatsache, dass die Bücher verstreut um das Bücherregal herum liegen. Stattdessen soll \enquote{Gegenstand} eine gewisse kleinstmögliche Einheit bezeichnen; so etwas wie ein Atom der Bedeutung. Als Nächstes gehe ich darauf genauer ein.

Schon frühmoderne Empiristen wie David Hume fragen sich, wie wir uns Dinge vorstellen können, die wir nie erfahren haben. Wie können wir uns beispielsweise einen goldenen Berg vorstellen, der bis zu den Wolken ragt, obwohl wir noch nie so einen Berg gesehen haben? Humes Lösung ist, dass wir Ideen, die wir erfahren (Berg, Gold, Wolken) auf neue Weisen miteinander kombinieren.\autocite[Vgl.][20]{hume1748} Wittgensteins Gedanke ähnelt dem. Wir können uns sprachlich auf so vieles und so verschiedenes beziehen und Sätze äußern, die wir noch nie zuvor gehört haben, weil wir eine bestimmte Menge an Gegenständen immer wieder neu miteinander kombinieren. Ein Gegenstand ist einfach die kleinstmögliche Einheit in so einem Kombinationsprozess. Es ist auch wichtig, zu bemerken, dass sich Gegenstände nicht beliebig kombinieren lassen. Jeder Gegenstand bestimmt, in welchen Sachverhalten er vorkommt und in welchen nicht. Insofern bestimmen die Gegenstände, was überhaupt vorstellbar ist.

Das ist noch etwas abstrakt, allerdings ist es auch schwierig, an dieser Stelle konkreter zu werden, denn Wittgenstein bringt kaum Beispiele für Gegenstände. Er behauptet zum Beispiel, dass sich die Gegenstände in einem \enquote{logischen Raum} befinden. Dieser logische Raum hat mehr Dimensionen als nur die drei, die wir aus dem Alltag gewohnt sind. So versteht Wittgenstein beispielsweise auch Farbe, Höhe eines Tons oder Härte einer Oberfläche als Dimensionen im logischen Raum. In dieser Metapher sind etwa Sachverhalte, die zwei Gegenstände verbinden, Geraden zwischen Punkten in diesen Dimensionen. Vielleicht sind Farbflecken, Tonhöhen, oder Tasteindrücke also konkretere Beispiele für Gegenstände.\footnote{Vgl. \cite{wittgenstein1922}, 2.013-2.0131.} Damit könnte es sein, dass Wittgenstein davon ausgeht, dass unsere Erfahrung aus vielen atomaren Sinneseindrücken besteht und dass diese die Gegenstände sind, die zu Tatsachen und Sachverhalten kombiniert werden. So würde er Ideen des logischen Empirismus voraus greifen.\footnote{Vgl. zu den Ideen des logischen Empirismus \cite{chalmers2007}, Kapitel 4: Der Induktivismus und \cite{schlick1934}.} Doch Wittgenstein sagt auch: \enquote{Beiläufig gesprochen: Die Gegenstände sind farblos.}\footnote{\cite{wittgenstein1922}, 2.0232.} Damit können wir Gegenstände zumindest nicht als Farbflecken verstehen. Vielleicht will er seine Konzeption von Gegenständen also abstrakt halten: Der Begriff bezeichnet die kleinsten Bausteine der Bedeutung, ganz gleich ob diese sich letztendlich als Sinneseindrücke, physikalische Elementarteilchen oder etwas ganz anderes herausstellen. Um die restlichen Aspekte von Wittgensteins Theorie anschaulicher zu erklären, werde ich hin und wieder dennoch alltägliche Beispiele von Gegenstände verwenden. Die LeserIn soll dabei nur im Auge behalten, dass Gegenstände im strengen Sinn abstrakter sind als in meinen Beispielen.

Soweit vorerst zu Wittgensteins Ontologie. Als nächstes werde ich erläutern, wie sich Wittgensteins Idealsprache auf die Wirklichkeit beziehen soll. Hier kommt auch endlich die abbildende Beziehung ins Spiel. Ein Bild besteht für Wittgenstein aus Elementen, die auf eine bestimmte Art und Weise angeordnet sind. Es bildet Sachverhalte und Tatsachen dadurch ab, dass die Art und Weise, auf die die Elemente im Bild angeordnet sind, die Art und Weise spiegelt, auf die Gegenstände in Sachverhalt oder Tatsache angeordnet sind. Betrachten wir zuerst \autoref{abb_foto}\footnote{\cite{mircea2019}.}.

\begin{figure}[h!]
  \centering
  \includegraphics[width=\linewidth]{assets/sängerin.jpg} % \linewidth
  \caption{Sängerin}
  \label{abb_foto}
\end{figure}

Im Vordergrund des Bilds ist eine Sängerin mit einem Mikrofon in der Hand zu sehen, im Hintergrund zwei Bandmitglieder und dahinter, verschwommen, etwas musikalisches Equipment. All das sind (vereinfacht ausgedrückt) Elemente des Bildes. Wenn der LeserIn diese Arbeit ausgedruckt vorliegt, dann sind sowohl die Abbildung der Sängerin als auch die des Mikrofons Tintenflächen, gedruckt auf Papier. Diese Tintenflächen stehen im Bild in bestimmten Beziehungen zu einander. Wittgenstein geht (grob gesagt) davon aus, dass das Bild die Tatsache abbildet, dass die Sängerin mit dem Mikrofon ein Lied singt, indem die Gegenstände dieser Tatsache analog zu den Tintenflächen im Bild angeordnet sind. Eine sprachliche Äußerung ist nach diesem Modell auch ein Bild. Sie besteht aus Elementen, die Gegenständen entsprechen und einer Art und Weise, diese Elemente anzuordnen, die einem Sachverhalt oder einer Tatsache entspricht. Bei sprachlichen Äußerungen nennt Wittgenstein die Elemente aber \enquote{Namen} und ihre Verbindung \enquote{Sätze}. Die Bedeutung eines Namens ist der ihm entsprechende Gegenstand, der Sinn eines Satzes besteht in seiner abbildenden Beziehung zu einem Sachverhalt.\footnote{Hieraus ergibt sich auch Wittgensteins Korrespondenztheorie der Wahrheit. Eine sprachliche Äußerung ist genau dann wahr, wenn sie einer Tatsache entspricht und sonst falsch. Mit dieser Korrespondenztheorie der Wahrheit können wir eine alternative Formulierung von Wittgensteins Kriterium sprachlicher Bedeutung geben: Eine Äußerung bedeutet die Bedingungen, unter denen sie wahr ist. Einen Satz zu verstehen, heißt einfach nur, zu wissen, unter welchen Umständen dieser Satz wahr wäre und wann nicht.}\footnote{Aus diesem Verständnis von Sätzen in Kombination mit Wittgensteins Verständnis von Wirklichkeit ergibt sich außerdem, was Wittgenstein zufolge Möglichkeit und Notwendigkeit sind. Dass etwas, das ein Satz behauptet, möglich ist, heißt einfach nur, dass ihm ein Sachverhalt (bzw. später eine Sachlage) entspricht. Da ein Satz erst dadurch einen Sinn hat, dass er mit einem Sachverhalt übereinstimmt, ist alles, das wir sinnvoll sagen können, möglich.}

Dass es für ein Bild so wichtig ist, in welchen Beziehungen seine Elemente stehen, zeigt sich an Wittgensteins Diskussion des Necker-Würfels.\footnote{Vgl. \cite{wittgenstein1922}, 5.5423.}\footnote{Diese Abbildung ist ein Auschnitt von \cite{lengerke2020}.}

\begin{figure}[h!]
  \centering
  \includegraphics[width=60mm]{assets/neckerwürfel_frei.png} % \linewidth
  \caption{Neckerwürfel}
  \label{neckerwürfel}
\end{figure}

Wir können diesen Würfel entweder so betrachten, dass uns die graue Fläche zugewandt ist, oder so dass sie uns abgewandt ist. Je nachdem betrachten wir den Würfel von unten oder von oben. Die Abbildung ist also ambig. Wittgenstein bemerkt, dass diese Ambiguität darin besteht, dass es verschiedene Arten gibt, aufzufassen, wie sich die Elemente des Bildes zu einander verhalten. Hier haben wir also einen weiteren Fall, der zeigt, dass es für eine abbildende Beziehung nicht nur auf die Elemente ankommt, sondern besonders auf die Art und Weise, auf die diese im Bild angeordnet sind.

Die Sätze, die ich bisher besprochen habe, setzen Namen in eine bestimmte Beziehung und stimmen genau mit einem Sachverhalt überein, der Gegenstände in eine analoge Beziehung setzt. Wittgenstein nennt solche Sätze \enquote{Elementarsätze}. Doch seine Gegenstände sind sehr grundlegend und abstrakt. Damit wir Sätze äußern können, die von Sängerinnen, Mikrophonen und Bücherregalen handeln, müssen wir Elementarsätze miteinander kombinieren. Das geschieht durch Wahrheitsfunktionen. Logische Junktoren wie \enquote{und}, \enquote{oder} und \enquote{wenn..., dann} sind manche dieser Wahrheitsfunktionen. Wahrheitsfunktionen bestimmen die Möglichkeiten, unter denen ein nach ihnen konstruierter komplexer Satz wahr ist. Denn sobald wir es mit mehr als einem Elementarsatz zu tun haben, werden auch mehr Möglichkeiten relevant. Für Elementarsätze gibt es je nur zwei: Der von ihnen abgebildete Sachverhalt kann bestehen oder auch nicht bestehen, der Satz kann wahr oder falsch sein. Bei komplexen Sätzen haben wir es aber mit mehr Möglichkeiten zu tun. Bei einem Satz, der aus den Elementarsätzen $p$ und $q$ besteht, gibt es beispielsweise vier Wahrheitsmöglichkeiten. Betrachten wir \autoref{tabelle_1}:

\begin{table}[h!]
\centering
\caption{Wahrheitsmöglichkeiten für 2 Elementarsätze}
\begin{tabular}{|l|l|l|}
\hline
1 & $p$ ist wahr   & $q$ ist wahr   \\ \hline
2 & $p$ ist wahr   & $q$ ist falsch \\ \hline
3 & $p$ ist falsch & $q$ ist wahr   \\ \hline
4 & $p$ ist falsch & $q$ ist falsch \\ \hline
\end{tabular}
\label{tabelle_1}
\end{table}

Wahrheitsfunktionen lassen sich dann dadurch definieren, unter welchen solcher Möglichkeiten ein Satz, der nach ihnen gebildet ist, wahr ist und wann er falsch ist. \autoref{tabelle_2} zeigt das für dieses Beispiel: % TODO: Checken, ob ich "Sachlage" richtig verwende

\begin{table}[h!]
\centering
\caption{Kombinationen von 2 Elementarsätzen}
\begin{tabular}{|l|l|l|l|}
\hline
  & $p \wedge q$ & $p \vee q$ & $p \subset q$ \\ \hline
1 & wahr        & wahr         & wahr               \\ \hline
2 & falsch      & wahr         & falsch             \\ \hline
3 & falsch      & wahr         & wahr               \\ \hline
4 & falsch      & falsch       & wahr               \\ \hline
\end{tabular}
\label{tabelle_2}
\end{table}

So eine Wahrheitsmöglichkeit ist eine Kombination von verschiedenen Sachverhalten, die bestehen oder nicht bestehen. Wittgenstein nennt sie daher manchmal auch \enquote{Sachlage}. Er sagt auch, dass ein Satz mit den Wahrheitsmöglichkeiten \enquote{übereinstimmt}, unter denen er wahr ist. % TODO: Stimmt das?

Der Sinn von Elementarsätzen ließ sich noch recht einfach charakterisieren: Er besteht in der Beziehung, durch die sie ihren Sachverhalt abbilden. Bei zusammengesetzten Sätzen wird diese abbildende Beziehung aber komplizierter. Ihr Sinn besteht in ihrem Übereinstimmen und nicht-Übereinstimmen mit Sachverhalten. Der Satz \enquote{$p$ und $q$} hat also einen anderen Sinn als der Satz \enquote{$p$ oder $q$}, weil der letztere mit allen Sachlagen außer 4 übereinstimmt, während der erstere nur mit Sachlage 1 übereinstimmt. Eine andere Art, dasselbe zu sagen wäre: Sprachliche Äußerungen haben dadurch, einen Sinn, dass sie manche Sachlagen ein- und andere ausschließen und sie unterscheiden sich darin, welche sie ein- beziehungsweise ausschließen.\footnote{Hier leitet Wittgenstein auch sein Verständnis von Wahrscheinlichkeit ab. Ein Satz behauptet etwas wahrscheinliches, wenn es besonders viele Sachlagen gibt, die mit ihm übereinstimmen. Vgl. \cite{wittgenstein1922}, 5.15-5.156.\label{fn_wahrscheinlichkeit}}\footnote{Das ist auch Wittgensteins Art, die heute immer noch populäre These zu artikulieren, dass die Bedeutung, bzw. der propositionale Gehalt eines Satzes in den Bedingungen besteht, unter denen er wahr ist.}

Logiker erforschen, wann manche dieser komplexen Sätze aus anderen folgen. Frege glaubt beispielsweise, dass wir logische Gesetze entdecken können, wahre Schlussregeln, die uns verraten, welche Sätze aus welchen anderen folgen. Indem Logiker solche Gesetze entdeckten, entdeckten sie notwendige Züge der Wirklichkeit. Doch Wittgenstein stellt sich entschieden gegen so ein Verständnis von Logik. Ihm zufolge kann es keine logischen Gesetze geben, weil alle Sätze der Logik dasselbe bedeuten. Denn die \enquote{Gesetze} der Logik sind Tautologien. Sie sind Arten und Weisen, Elementarsätze so zu kombinieren, dass die resultierenden Sätze bei allen Wahrheitsmöglichkeiten wahr sind. Da Sinn aber nach Wittgenstein erst dadurch zustande kommt, dass ein Satz bei bestimmten Sachlagen wahr ist, während er bei anderen falsch ist, sind logische Gesetze in Wittgensteins Worten \enquote{sinnlos}.

Ich habe also geschildert, was Wittgenstein zufolge die Wirklichkeit ist und wie diese strukturiert ist und ich habe seine Idealsprache bestehend aus Namen, Elementarsätzen und Wahrheitsfunktionen erläutert. Im Zuge dessen bin ich auch darauf eingegangen, wie diese Sprache sich auf die Wirklichkeit bezieht. Hier ist der Großteil von Wittgensteins Arbeit getan. Denn er geht davon aus, dass nur das, was sich in seiner Idealsprache formulieren lässt, sinnvoll (bzw. sinnlos) ist, alles andere ist Unsinn. Um es metaphorisch auszudrücken: Wittgensteins Idealsprache markiert eine Grenze von Sprache mit Sinn auf der einen- und Unsinn auf der anderen Seite. Die Logik liegt genau auf dieser Grenze, sie ist sinnlos. Meine Frage nach den Grenzen unserer Sprache ist also beinahe beantwortet. Im Folgenden werde ich diese Antwort genauer ausformulieren, indem ich erkläre, inwiefern Wittgenstein in ihr eine Lösung der philosophischen Probleme sieht. Danach werde ich untersuchen, ob Wittgensteins Antwort stichhaltig ist.

\subsection{Wie Wittgenstein die philosophischen Probleme löst}

% TODO: Hier kurz Hackers möglichen Einwand vorwegnehmen: Schlecht belesener Wittgenstein?

Kommen wir also zu Wittgensteins Lösungen der philosophischen Probleme. Ich habe oben geschildert, dass Wittgenstein die gesamte Philosophie, so wie wir sie heute verstehen, lösen will. Diese Disziplin ist ein weites Feld, ich werde also zuerst eine grobe Landkarte der Philosophie zeichnen und dann kurz auf Wittgensteins Lösungen der jeweiligen Probleme eingehen.

Im Allgemeinen lässt sich die Philosophie in praktische und theoretische Philosophie aufteilen. Die praktische Philosophie beschäftigt sich damit, wie wir leben sollen. Dazu gehört zum einen die normative Ethik, die untersucht, was die Moral uns abverlangt, zum anderen gehört dazu aber auch die Staatstheorie, in der untersucht wird, wie wir unsere Gesellschaft organisieren sollen. Im Rahmen der praktischen Philosophie werden dann auch verschiedene konkrete praktische Probleme diskutiert, zum Beispiel die Rolle von Geschlecht, Hautfarbe und Sexualität in unserer Gesellschaft. Auch Metaethik, also den Versuch zu verstehen, wie der ethische Diskurs aufzufassen ist, könnte man noch zur praktischen Philosophie rechnen.

Während die praktische Philosophie sich eher damit beschäftigt, wie die Dinge sein sollen, geht es in der theoretischen eher darum, wie sie sind. Hierzu gehört die Ontologie, die versucht herauszufinden, was es gibt und wie das, was es gibt, strukturiert ist, aber auch die Erkenntnistheorie, die erforscht, was Wissen und Wahrheit sind und wie wir zu diesen kommen. Zur theoretischen Philosophie gehören außerdem noch die Philosophie des Geistes und die Sprachphilosophie.

Die Ästhetik ist eine philosophische Disziplin, die nicht klar in diese Aufteilung passt. Sie erforscht unsere Kunstpraktiken und fragt sich unter anderem, was es heißt, dass etwas ein Kunstwerk ist. Außerdem gibt es unzählige \emph{Philosophien der...}, die Überlegungen aus den oben geschilderten Bereichen im Rahmen einer Reflexion anderer Disziplinen anstellen. Beispiele hierzu wären die Wissenschaftstheorie, die Philosophie der Logik und Mathematik oder die Philosophie des Rechts. Wenn Wittgenstein beansprucht, die philosophischen \enquote{Probleme im Wesentlichen endgültig gelöst zu haben}, dann geht es ihm um all diese Disziplinen.

Betrachten wir zuerst Wittgensteins Lösungen der Probleme der praktischen Philosophie. Diese handelt er mit einigen wenigen Überlegungen ab, durch die er zeigt, dass seine Idealsprache keine Normativität einfangen kann. Da er seine Idealsprache als Kriterium für sinnvolle Sprache verwendet, sind also jegliche normativen Äußerungen unsinnig und daher zurückzuweisen. Der Grundgedanke ist, dass in seiner Ontologie keine moralischen Werte vorkommen:

\begin{quote}
  Der Sinn der Welt muß außerhalb ihrer liegen. In der Welt ist alles, wie es ist, und geschieht alles, wie es geschieht; es gibt in ihr keinen Wert -- und wenn es ihn gäbe, so hätte er keinen Wert.

  Wenn es einen Wert gibt, der Wert hat, so muß er außerhalb alles Geschehens und So-Seins liegen. Denn alles Geschehen und So-Sein ist zufällig.

  Was es nichtzufällig macht, kann nicht in der Welt liegen, denn sonst wäre dies wieder zufällig.

  Es muß außerhalb der Welt liegen.\footnote{\cite{wittgenstein1922}, 6.41.}
\end{quote}

Weder Sachverhalte noch Sachlagen sind in irgendeiner Form gut oder schlecht. Denn dass etwas gut oder schlecht ist, muss nochmal etwas ganz anderes sein, als dass bestimmte Gegenstände auf eine Art und Weise verbunden werden. Wittgensteins Idealsprache kann also nicht ausdrücken, dass eine Handlung gut oder schlecht ist und ein Versuch, so etwas zu tun, ist unsinnig. Das klärt dann schon Wittgensteins Metaethik: Unser praktischer Diskurs ist unsinnig. Er besteht in Versuchen, zu sagen, was unaussprechlich ist. Und auch Ethik, Staatstheorie und die verschiedenen konkreten Überlegungen über unsere gesellschaftliche Praxis sind damit abgehandelt. Wir sollten an ihrer Stelle lieber schweigen. Jetzt kann man darüber streiten, ob diese Einschätzung plausibel ist\footnote{\cite{dain2014} vertritt die These, dass unser ethischer Diskurs unsinnig ist, beispielsweise noch heute. Entgegen dessen könnte man argumentieren, dass Wittgensteins Formalismus, gerade weil er unseren praktischen Diskurs nicht einfangen kann, zurückzuweisen ist. Schließlich gibt es heute mit der Deontischen Logik auch Ansätze, denen das gelingt. Vgl. zu denotischer Logik \cite{garson2018}, Abschnitt 3 - Deontic Logics.}, darauf kommt es hier aber nicht an. Mir geht es hier nur darum, meine Interpretation zu rechtfertigen, dass Wittgensteins TLP \emph{mit dem Anspruch} auftritt, die Philosophie qua Philosophie als unsinnig aufzuweisen.\footnote{Es gibt auch eine Debatte darüber, inwiefern der TLP eine konkrete Ethik entwickelt, die manches gegenüber anderem auf- beziehungsweise abwertet. Ich kann diese Debatte in diesem Rahmen vernachlässigen, weil alle Beteiligten sich einig sind, dass ethische Forderungen dem TLP zufolge unsinnig sind. Die Frage ist nur, wie diese Unsinnigkeit genau zu verstehen ist und was diese unsinnigen Sätze für unser Leben bedeuten. Vgl. z.B. \cite{diamond2000} für die Interpretation dass Wittgensteins Ethik dadurch funktioniert, dass jemand mit bösem Willen in seinem Herzen etwas unsinniges sagt; \cite{cahill2017} für die Interpretation, dass Wittgensteins Unsinn eine ethische Forderung ist, zu staunen.} Eng damit verwandt ist Wittgensteins Lösung der Probleme der Ästhetik. Hierzu schreibt Wittgenstein nur in Klammern einen Satz, nachdem er uns verrät, dass \enquote{sich die Ethik nicht aussprechen läßt}: \enquote{Ethik und Ästhetik sind Eins.} Wittgenstein scheint also anzunehmen, dass auch die Ästhetik vorwiegend in normativen Überlegungen besteht, die sich in seiner Idealsprache nicht ausdrücken lassen und daher unsinnig sind.\footnote{Vgl. \cite{wittgenstein1922}, 6.421.}

Wittgensteins Lösungen der Probleme der theoretischen Philosophie sind komplexer. Auf seine Ontologie und seine Sprachphilosophie bin ich oben schon eingegangen. Es gibt ausschließlich Gegenstände, Sachverhalte, usw. und sprachliche Äußerungen beziehen sich auf diese, indem sie sie abbilden. Das heißt auch, dass es z.B. Kausalität nicht gibt:

\begin{quote}
  5.135   Auf keine Weise kann aus dem Bestehen irgendeiner Sachlage auf das Bestehen einer von ihr gänzlich verschiedenen Sachlage geschlossen werden.

  5.136   Einen Kausalnexus, der einen solchen Schluß rechtfertigte, gibt es nicht.

  5.1361  Die Ereignisse der Zukunft können wir nicht aus den gegenwärtigen erschließen.
  Der Glaube an den Kausalnexus ist der Aberglaube.\footnote{\cite{wittgenstein1922}, 5.135-5.1361.}
\end{quote}

In derselben Überlegung handelt Wittgenstein Humes erkenntnistheoretisches Induktionsproblem ab. Mit seinem Induktionsproblem bemerkt Hume, dass wir aufgrund von vergangenen Ereignissen auf zukünftige schließen, indem wir kausale Zusammenhänge annehmen. So ein Schluss setzt voraus, dass die Zukunft auch so sein wird wie die Vergangenheit. Diese Annahme können wir aber nicht wissen, da sie einerseits keine begriffliche Wahrheit ist und weil es andererseits zirkulär wäre, sie durch empirische Beobachtung zu stützen.\autocite[][Kapitel 4: Skeptische Zweifel in Betreff der Verstandestätigkeiten]{hume1748} Wittgenstein ergänzt dieses Problem durch ein genaueres Verständnis dessen, was es hier heißt, dass etwas ein Grund ist und dass etwas erschlossen wird. Einen Grund für einen Satz anzuführen, heißt nur, einen weiteren Satz zu äußern, der mit einer Wahrheitsmöglichkeit des ersteren übereinstimmt.\footnote{Vgl. für Wittgensteins Verständnis von Gründen \cite{wittgenstein1922}, 5.101.} Und in diesem Sinn kann ein Satz, der das Bestehen einer Sachlage behauptet, weil er von der Gegenwart handelt, schlicht kein Grund sein für einen anderen Satz, der das bestehen einer anderen, zukünftigen Sachlage behauptet.

Das alles ist natürlich nur ein grober Einblick in Wittgensteins umfassendes Projekt. Er sollte aber ausreichen, um einen Eindruck davon zu vermitteln, was Wittgenstein meint, wenn er sich vornimmt, \enquote{die philosophischen Probleme} zu \enquote{lösen}. Wittgenstein konstruiert im TLP eine formale Sprache. Sobald wir philosophische Fragen in diese formale Sprache übersetzen, sollen wir erkennen, dass diese falsch gestellt waren. Rede von Gutem und Schlechtem oder von kausalen Zusammenhängen ist unsinnig und zurückzuweisen und damit sind die Fragen danach, was das ist, auch geklärt. Als nächstes werde ich untersuchen, ob diese Art, die Grenzen unserer Sprache zu ziehen, plausibel ist.

% Legacy Notiz: Was nach Wittgenstein noch alles unsinnig ist:
% ============================================================
% Kunst, Sinne und Mystik
% Ästhetik, Ethik und Metaphysik
% Existenzialistisches Gerede (Besonders über das Sein und das Nichts)
% Metaethische Debatten über moralische Tatsachen
% Manche Hypothesen mit keinerlei empirischen Konsequenzen (Da ist ein Geist in meiner Flasche, Wir leben in einer Simulation, die Welt ist erst vor ein paar Sekunden entstanden, die Realität ist nur ein Traum, Gott hat die Welt geschaffen)

\section{Leiter-Probleme}

% TODO: Hier noch ein kurzer Abschnitt über alles, das aus den Grenzen von Sprache rausfällt?

% Formale Begriffe

\subsection{Sagen und Zeigen}\label{sagen_und_zeigen}

Wie wir gesehen haben, zieht Wittgensteins Bildtheorie extrem enge Grenzen von Sprache. Allein das macht sie schon kontrovers und rätselhaft. Besonders rätselhaft wird sie aber dadurch, dass sie selbst außerhalb der Grenzen liegt, die sie zieht.

Alltägliche Sätze und Sätze in der Naturwissenschaft haben nach Wittgenstein Sinn, indem sie Sachverhalte abbilden. Sie sind durch Wahrheitsfunktionen aus Elementarsätzen zusammengesetzt, die jeweils mit dem Bestehen von einem Sachverhalt übereinstimmen. Doch was ist mit Sätzen, die von Sachverhalten, Elementarsätzen, Gegenständen und Tatsachen handeln?

Ein Satz wie \enquote{Die Welt zerfällt in Tatsachen}\footnote{\cite{wittgenstein1922}, 1.2.} wirkt, oberflächlich betrachtet, wie ein gewöhnlicher Satz. Nach Wittgensteins Theorie könnte er ein Elementarsatz sein. Dann wären \enquote{Welt} und \enquote{Tatsachen} Namen für Gegenstände und der Satz würde mit dem Sachverhalt übereinstimmen, dass ersteres in letzteres zerfällt. Aber das ist offensichtlich Unsinn. Wenn wir den Satz so verstehen, dann ist es Wittgenstein nicht gelungen, sich durch das Wort \enquote{Welt} auf die Welt und durch das Wort \enquote{Tatsachen} auf alle Tatsachen zu beziehen. Schließlich sind die Welt und die Tatsachen, in die sie zerfällt, etwas ganz anderes als die Gegenstände, aus denen sich die Tatsachen zusammensetzen! Hier hilft es auch nicht, den Satz in kleinere Elementarsätze zu zerlegen. Die Grundgedanken von Wittgensteins Bildtheorie der Sprache lassen sich also, Wittgensteins Bildtheorie der Sprache zufolge, nicht ausdrücken.

Wittgenstein beschreibt das folgendermaßen:

\begin{quote}
  4.12 Der Satz kann die gesamte Wirklichkeit darstellen, aber er kann nicht das darstellen, was er mit der Wirklichkeit gemein haben muß, um sie darstellen zu können -- die logische Form. [..]

  4.121 [..] Der Satz \emph{zeigt} die logische Form der Wirklichkeit. Er weist sie auf. [..]

  4.1212 Was gezeigt werden \emph{kann}, \emph{kann} nicht gesagt werden.\footnote{\cite{wittgenstein1922}, 4.12-4.1212.}
\end{quote}

Wir begegnen hier demselben Problem, das auch Freges und Russells Ansätze plagt. Die philosophische Idealsprache der Autoren kann über vieles reden, aber nicht über sich selbst. Freges und Russells Ansätze scheitern an diesem Problem noch nicht, da sie ihre Idealsprachen als ein Mittel neben vielen anderen verstehen. Für Wittgenstein ist das Problem ernster. Denn seine Idealsprache soll die Grenze sprachlichen Ausdrucks schlechthin ziehen. Wenn sie jenseits von dieser liegt, widerlegt sein Ansatz sich selbst.

In der Textstelle oben sehen wir, dass sich Wittgenstein dieses Problems bewusst ist und dass er eine Lösung anbietet. Ich werde Wittgensteins Lösung in diesem Abschnitt zunächst nur vorläufig rekonstruieren. Denn aus der Lösung ergeben sich Probleme, die wiederum neue Interpretationen erfordern, die ich später besprechen werde.

Verstehen wir Wittgensteins Lösung damit vorläufig so, dass seine Idealsprache nicht \emph{gesagt}, sondern \emph{gezeigt} wird. Da sie nur die Grenzen des \emph{Sagbaren} zieht, überschreitet sie diese, indem sie \emph{gezeigt} wird, also nicht selbst. Wittgenstein scheint damit zu behaupten, dass jeder Satz zwei Inhalte kommuniziert. Erstens den, den er sagt. Dieser ist von Satz zu Satz verschieden, durch ihn tauschen wir Ideen aus und koordinieren uns untereinander. Und zweitens, den, den er zeigt. Dieser ist bei allen Sätzen gleich: Er besagt, dass zwischen Sätzen und Wirklichkeit eine Beziehung der logischen Abbildung besteht. Oder anders ausgedrückt: Er kommuniziert die Struktur der Wittgensteinschen Idealsprache. Wittgenstein versucht dann mit dem TLP die philosophischen Probleme zu lösen, indem er uns durch unsinnige Äußerungen darauf aufmerksam macht, was die Sätze unserer Alltagssprache zeigen. So sollen wir lernen, zwischen unsinnigen Sätzen, z.B. der Ethik und sinnvollen Sätzen, z.B. der Naturwissenschaft zu unterscheiden. Aus diesen Überlegungen ergibt sich dann der rätselhafte Schluss des TLP. Da er von zentraler Relevanz für die weitere Diskussion ist, werde ich ihn im Folgenden abgekürzt als \textbf{Leiter-Satz} bezeichnen:

\begin{quote}
  6.54 Meine Sätze erläutern dadurch, daß sie der, welcher mich versteht, am Ende als unsinnig erkennt, wenn er durch sie -- auf ihnen -- über sie hinausgestiegen ist. (Er muß sozusagen die Leiter wegwerfen, nachdem er auf ihr hinaufgestiegen ist.)

  Er muß diese Sätze überwinden, dann sieht er die Welt richtig.

  7 Wovon man nicht sprechen kann, darüber muß man schweigen.\footnote{\cite{wittgenstein1922}, 6.54-7.}
\end{quote}

Die Sätze des TLP sind also unsinnig, weil sie nichts \emph{sagen}. Sie helfen uns nur, zu erkennen, was andere Sätze \emph{zeigen}. In diesem Sinn soll die LeserIn sie überwinden. Diese Überlegungen werfen einen Komplex von Problemen auf, den ich \enquote{die Leiter-Probleme} nenne. Sie werde ich im nächsten Abschnitt schildern.

\subsection{Die Leiter-Probleme}

Beginnen wir mit dem ersten Problem: \emph{Wenn Wittgenstein andere Philosophie aufgrund ihrer Unsinnigkeit ablehnt, warum ist seine dann nicht abzulehnen?} Schließlich war es ja die Mission des TLP, die philosophischen Probleme im Wesentlichen endgültig zu lösen. Diese \enquote{Lösung} bestand dann darin, dass sowohl philosophische Fragestellungen als auch klassische Antworten auf philosophische Fragen als unsinnig aufgewiesen wurden. Damit dieser Nachweis als Lösung verstanden werden kann, muss Unsinnigkeit philosophische Positionen aber widerlegen. Unsinnigkeit muss schlecht sein, schlimm genug, um eine Position beiseite zu schieben und sie nicht weiter zu diskutieren. Ansonsten scheitert Wittgensteins Projekt, denn es gelingt ihm dann nicht, die restliche Philosophie beiseite zu schieben. Gleichzeitig praktiziert Wittgenstein aber selbst Philosophie. Er behandelt philosophische Fragestellungen wie \enquote{Was ist die Welt?} und formuliert Antworten wie \enquote{Die Welt ist die Gesamtheit der Tatsachen, nicht der Dinge.}\footnote{\cite{wittgenstein1922}, 1.1.} Und im \textbf{Leiter-Satz} sagt Wittgenstein dann, dass \enquote{der welcher [ihn] versteht}, seine Sätze \enquote{am Ende als unsinnig erkennt}. Damit müssten wir ihm zufolge aber auch verstehen, dass seine Sätze sich selbst disqualifizieren! Der TLP müsste also seiner eigenen Kritik zum Opfer fallen und sich selbst widerlegen. Nennen wir dieses Problem \textbf{Das Problem der Selbstwiderlegung des Philosophie-Skeptizismus}.\footnote{Eine Position ist skeptisch, wenn sie bestreitet, dass wir in einem bestimmten Bereich Wissen haben. Verstehen wir Wissen als wahre gerechtfertigte Überzeugung, bietet das SkeptikerInnen verschiedene Angriffspunkte. Der klassische Skeptizismus ist erkenntnistheoretisch: Laut ihm haben wir kein Wissen, weil Rechtfertigungen fehlen. Hierzu gehört zum Beispiel der Außenweltskeptizismus René Descartes: Unsere Wahrnehmungen sind keine geeignete Rechtfertigung, um auf eine Außenwelt zu schließen, da sie auch ohne diese bestehen könnten. Vgl. \cite{descartes1954}. Wittgensteins Skeptizismus ist dagegen konstitutiv. Konstitutiver Skeptizismus setzt an der Wahrheit unseres vermeintlichen Wissens an. So bestreitet beispielsweise Willard Van Orman Quine, dass es Bedeutungen gibt, woraus folgt, dass wir auch kein Wissen von Bedeutungen haben können. Vgl. u.a. \cite{quine1976} und \cite{quine1969}, Kapitel 2: Ontological Relativity. Ähnlich bestreitet Wittgenstein, dass es philosophische Tatsachen gibt, woraus folgt, dass wir kein philosophisches Wissen haben können.}

Dieses Problem ergibt sich aus Wittgensteins ambitioniertem Anspruch, die Philosophie qua Philosophie als unsinnig aufzuzeigen und ihre Fragestellungen so zurückzuweisen. Ich habe oben gezeigt, dass Wittgenstein diesen Anspruch vertritt. Für die Frage, die ich hier diskutiere, also wie man die Grenzen unserer Sprache verstehen kann, ist er aber nicht unbedingt notwendig. Es könnte ja schließlich sein, dass Wittgenstein die Grenzen unserer Sprache erkannt hat, diese aber leider wenig damit zu tun haben, wie philosophische Probleme zu lösen sind. Wir könnten Wittgensteins Projekt also um den ambitionierten Anspruch kürzen und so eine zentrale Prämisse des obigen Problems zurückweisen. Doch dann stehen wir trotzdem vor weiteren Problemen.

Wittgenstein zufolge sind die Einsichten, die der TLP vermittelt, nicht sagbar. Sie zeigen sich aber in manchen Sätzen. Hier stellt sich die Frage, wie dieses Zeigen zu verstehen ist. Bisher habe ich Wittgenstein so interpretiert, dass sinnvolle (und sinnlose) Sätze, neben dem Inhalt den sie sagen, einen weiteren Inhalt zeigen. Wenn das aber so ist, dann drängt sich die Frage auf, ob auch das Zeigen in irgendeiner Form begrenzt ist. Die Frage, mit der ich diese Abhandlung begonnen habe, war, ob und wie wir die Grenzen unserer Sprache verstehen können. Doch wenn das, was wir mit Sprache \emph{sagen} können, durch Wittgensteins Idealsprache begrenzt ist, das was wir mit Sprache \emph{zeigen} können aber nicht, dann ist es Wittgenstein gar nicht gelungen, eine Grenze unserer Sprache zu ziehen! Schließlich könnte man die Grenze des Sagbaren dann immernoch überschreiten, indem man etwas zeigt. Auch Wittgenstein Philosophiekritik wäre so schwächer. PhilosophInnen, die vermeintlich Unsinn reden, könnten sich einfach darauf berufen, dass sie etwas zeigen, dass sich bloß nicht sagen lässt. Nennen wir dieses Problem \textbf{Das Problem der Grenzen des Zeigbaren}.

Ein weiteres Problem ergibt sich, wenn wir uns fragen, ob schon die Sätze des TLP selbst den Aufbau von Wittgensteins Idealsprache zeigen, oder ob sich dieser nur in Sätzen unserer Alltagssprache, etc. zeigt. Nehmen wir zuerst einmal an, dass Wittgensteins Unsinn die Struktur seiner Idealsprache zeigen soll. Hier entfernen wir uns schon etwas vom Primärtext, denn im TLP schreibt Wittgenstein nie, dass sein Unsinn etwas zeigt. Er schreibt nur explizit, dass Sätze auf die wir durch den Unsinn aufmerksam gemacht werden, etwas zeigen. Doch das schließt diese Interpretation nicht aus, schließlich widerspricht er ihr nicht explizit. Die nächste Frage hier wäre aber, wie sowohl unsinnige- als auch sinnlose und sinnvolle Sätze auf dieselbe Weise zeigen können. Schließlich versuchen wir \enquote{zeigen} gerade als Fachbegriff zu verstehen, der kohärent immer auf dieselbe Weise verwendet wird. Bleiben wir dazu zuerst beim unkontroversen Fall: Elementarsätzen in Wittgensteins Idealsprache. In diesen zeigt sich etwas dadurch, dass sie übersichtlich strukturiert sind. Im Satz \enquote{$aRb$} zeigt sich beispielsweise, dass die Gegenstände $a$ und $b$ im Sachverhalt $R$ verbunden sind. Denn dieser Satz hat eine übersichtliche Struktur, die der Struktur des Sachverhalts entspricht. Der Satz \enquote{Der Sachverhalt ist eine Verbindung von Gegenständen.} hat aber keine solche Struktur. Dieser Satz ist unsinnig, denn er entspricht keinem Sachverhalt und keiner Sachlage. Damit kann er also zumindest nicht in der Hinsicht zeigen, dass er sich seine Struktur, mit dem, mit dem er übereinstimmt, teilt. Er ist auch nicht auf dieselbe Art übersichtlich wie \enquote{$aRb$}. Denn auch diese Übersichtlichkeit besteht ja wieder darin, dass die Struktur des Sachverhalts, mit dem er übereinstimmt, eine Struktur, bei der zwei Elemente auf eine Weise verbunden werden, hier durch eine visuelle Komposition offensichtlich dargestellt wird. Auch das kann beim unsinnigen Satz nicht funktionieren, weil dieser ja mit keinem Sachverhalt übereinstimmt. Generell müssen wir also feststellen, dass \enquote{zeigen} bei sinnvollen, beziehungsweise sinnlosen im Gegensatz zu unsinnigen Sätzen unterschiedlich verstanden werden muss. Da Wittgenstein \enquote{zeigen} explizit nur auf sinnvolle und sinnlose Sätze bezieht, ist es also unplausibel, denselben Begriff auch für unsinnige Sätze zu verwenden. Hier ergibt sich aber ein weiteres Leiter-Problem. Denn selbst wenn wir die bisher geschilderten Probleme außen vor lassen, kann Wittgenstein jetzt immer noch nicht erklären, wie der TLP die Erkenntnisse, die er uns mitteilen will, überhaupt kommunizieren kann. Schließlich sind die einzigen beiden Arten, auf die ein Satz nach Wittgenstein einen Inhalt kommunizieren kann, ihn zu sagen und ihn zu zeigen. Doch die Sätze des TLP tun keins von beidem! Dieses Problem werde ich \textbf{das Problem der Kommunikation des TLP} nennen.\footnote{An dieser Stelle ist auch zu bemerken, dass Wittgenstein es PhilosophInnen, die sich jenseits der Grenzen des Sagbaren bewegen, hier noch leichter machen muss. Sie müssen jetzt keine sinnvollen Sätze mehr formulieren, und sie müssen nicht einmal Sätze formulieren, die etwas zeigen. Sie können sich einfach darauf berufen, dass sie uns darauf hinweisen, dass andere Sätze etwas zeigen.}

Ein viertes Problem ist das folgende. Wenn Wittgensteins TLP unsinnig ist, kann er keine Argumente enthalten. Schließlich setzen Argumente eine logische Struktur voraus. Und Wittgenstein bestreitet ja, dass Unsinn eine solche Struktur haben kann. Wir sollten philosophische Positionen aber nur akzeptieren, wenn gute Gründe für sie sprechen. Da diese Gründe in Argumenten formuliert sein müssten, ist Wittgensteins TLP also zurückzuweisen.

Argumente lassen sich grob in induktive und deduktive unterteilen.\autocite[Vgl. für die diesem Argument zugrundeliegende Argumentationstheorie][]{sinnott-armstrong2014} Beide bestehen aus einer Menge von Prämissen, aufgrund von denen auf eine Konklusion geschlossen wird. Bei einem deduktiven Argument ist dieser Schluss erlaubt, weil die Konklusion logisch aus den Prämissen folgt, was bedeutet, dass bei wahren Prämissen die Konklusion notwendigerweise auch wahr ist. Das muss bei induktiven Argumenten aber nicht der Fall sein. Hier reicht es, wenn die Prämissen gute Evidenzen für die Konklusion sind, und diese zumindest wahrscheinlich machen. Ein klassisches Beispiel für ein deduktives Argument ist:

\begin{enumerate}
  \item Wenn es regnet, ist die Straße nass.
  \item Es regnet.
  \item \textbf{Also:} Die Straße ist nass.
\end{enumerate}

Wenn \enquote{Es regnet.} und \enquote{Die Straße ist nass.} Elementarsätze wären, die ich der Einfachheit halber hier durch $R$ und $N$ darstelle, wäre dieses Argument in Wittgensteins Idealsprache:

\begin{enumerate}
  \item $R \subset N$
  \item $R$
  \item \textbf{Also:} $N$
\end{enumerate}

Dieses Argument ist also nur gültig und damit nur akzeptabel, weil wir in der logischen Idealsprache erkennen, dass \enquote{$R \subset N \wedge R \rightarrow N$} eine sinnlose Tautologie ist. Dass Sätze unsinnig sind, heißt aber gerade, dass sie sich nicht in diese Idealsprache übersetzen lassen. Bei Argumenten, die aus unsinnigen Sätzen bestehen, lässt sich also keine Gültigkeit feststellen. Damit lassen sich in unsinnigen Sätzen also auch keine akzeptablen deduktiven Argumente formulieren.

Wittgenstein scheint aber ohnehin nur an wenigen Stellen deduktiv zu argumentieren.\footnote{So zum Beispiel in seinem Argument dafür, dass das Negationszeichen keinen Gegenstand vertritt oder seinem Argument dafür, dass eine Liste mit allen Gegenständen noch keine Beschreibung der Welt ist.} Oft tätigt er einfach Aussagen, ohne dass offensichtlich ist, wie diese begründet werden. Das merkt man beispielsweise schon an den ersten Sätzen des TLP. Wittgenstein stellt zuerst in den Raum, dass \enquote{Die Welt [alles ist], was der Fall ist.} Die folgenden Sätze begründen diese Behauptung nicht, sondern führen sie nur weiter aus:

\begin{quote}
  1.11	Die Welt ist durch die Tatsachen bestimmt und dadurch, daß es alle Tatsachen sind.

  1.12	Denn, die Gesamtheit der Tatsachen bestimmt, was der Fall ist und auch, was alles nicht der Fall ist.
\end{quote}

Dieser Stil könnte bei unaufmerksamen LeserInnen den Anschein erwecken, dass Wittgenstein dogmatisch ist, oder dass er nicht mehr tut, als ein eigenwilliges Vokabular zu definieren. Doch es gibt eine wohlwollendere Interpretation: Vielleicht ist der gesamte TLP ein einziger Schluss auf die beste Erklärung.

Schlüsse auf die beste Erklärung fallen unter die induktiven Argumente. Da Wittgenstein Wahrscheinlichkeit in seiner Idealsprache einfängt, können wir vermutlich annehmen, dass er damit einverstanden wäre, induktive Argumente als solche zu verstehen, bei denen die Konklusion gegeben der Prämissen wahrscheinlich ist.\footnote{Vgl. \cite{wittgenstein1922}, 5.15-5.156 für Wittgensteins Verständnis von Wahrscheinlichkeit. Ich erkläre dieses Verständnis außerdem in Fußnote \cref{fn_wahrscheinlichkeit}.}

Bei Schlüssen auf die beste Erklärung wird aufgrund von generellen Prinizipien wie Einfachheit und Erklärungskraft eine Erklärung von vielen als die beste für ein bestimmtes Phänomen ausgewählt. Diese wird dann mit all ihren Implikationen akzeptiert. Zu den Prämissen eines Schlusses auf die beste Erklärung gehören also eine Phänomenbeschreibung und eine Annahme, dass eine bestimmte Erklärung dieses Phänomen am besten erklärt. Die Konklusion ist dann die Erklärung. Hier ist schon fragwürdig, ob ein solches Verständnis von Argumentation in Wittgensteins Philosophie passt. Schließlich müssten die generellen Prinzipien mit wahren oder wahrscheinlichen Aussagen begründet werden, die letztendlich auf Elementarsätze zurückgehen. Das lässt sich aber schon nicht umsetzen, weil ein Prinzip dafür, welche Erklärung \enquote{besser} als eine andere ist, Normativität voraussetzt, die es Wittgenstein zufolge nicht gibt. So ist beispielsweise auch ein Prinzip, das besagt, dass einfachere Erklärungen besser sind, ein Werturteil. Doch selbst wenn wir dieses Problem außen vor lassen, hat die Interpretation von Wittgensteins TLP als Schluss auf die beste Erklärung grundlegende Schwierigkeiten.

Betrachten wir, um diese zu illustrieren, ein Beispiel für einen Schluss auf die beste Erklärung: den teleologischen Gottesbeweis. Ihm zufolge ist unsere Welt zweckmäßig strukturiert. Das ist das Phänomen. Früchte wachsen an Bäumen, damit wir essen. Löwen haben lange Zähne, um ihre Beute zu reißen. Fische haben passend zu ihren Flossen Kiemen, um unter Wasser zu atmen und zu schwimmen, während Menschen Lungen haben, die für die Luft bestimmt sind. Dass das alles nur zufällig so ist, ist keine gute Erklärung. Schließlich wäre der Zufall bei der zweckmäßigen Struktur menschlicher Artefakte auch keine gute Erklärung. Die zweckmäßigen Strukturen unserer Welt lassen sich am besten dadurch erklären, dass jemand all diese Dinge mit ihren jeweiligen Zwecken im Sinn geschaffen hat. Es gibt also Gott als Schöpfer dieser Welt.\footnote{Es geht mir hier nicht um die Plausibilität des Arguments. Diese wird nicht nur durch David Humes Kritik, sondern besonders durch Charles Darwins Beobachtung, dass die scheinbar zweckmäßigen Strukturen der Natur auf evolutionäre Prozesse zurückgehen, die ohne Schöpfer auskommmen, vernichtend unterminiert. Das Argument dient hier nur als einfache Illustration eines Schlusses auf die beste Erklärung und in diesem Sinn auch als Kontrast zu Wittgensteins philosophischer Methode. Vgl. für eine einfache Erklärung der Evolutionstheorie \cite{dawkins1989} und für eine Diskussion des teleologischen Gottesbeweises \cite{bromand2011}, Einleitung zu Teil II: Die Neuformulierung der Gottesbeweise in der frühen Neuzeit; \cite{bromand2011}, Kapitel 9: David Hume: Die empiristische Kritik des teleologischen Gottesbeweises.}

Vielleicht will Wittgenstein mit dem TLP ähnlich argumentieren. Der gesamte TLP ist eine Erklärung für Phänomene der Sprache, Referenz, Wahrheit, etc. Dass es sich um die beste Erklärung handelt, sollen wir im Vergleich mit anderen Erklärungen wie denen von Frege und Russell merken. Auf den ersten Blick vermeidet eine solche Interpretation das Problem, dass Wittgensteins Unsinn keine logische Struktur aufweisen kann, weil innerhalb der Erklärung keine Argumente mit einer logischen Struktur formuliert werden müssen. So ist die Schöpfungsgeschichte in sich kein Argument und muss, um ihre Rolle im teleologischen Gottesbeweis spielen zu können, auch noch keins sein. Auf den zweiten Blick zeigt sich aber, dass Schlüsse auf die beste Erklärung an anderer Stelle logische Beziehungen erfordern, die Wittgenstein nicht zulassen darf. Denn die Phänomenbeschreibung muss in Verbindung mit der Erklärung gesetzt werden und das geschieht durch logische Folgerung. Damit \enquote{Ein allmächtiges Wesen hat alles in der Welt mit bestimmten Zwecken im Sinn geschaffen.} eine Erklärung für \enquote{Unsere Welt ist zweckmäßig strukturiert.} ist, muss letzteres aus ersterem folgen. Auf derartige Folgerungsbeziehungen darf sich Wittgenstein mit einer unsinnigen Argumentation aber, wie ich es eben im Kontext von deduktiven Argumenten erklärt habe, nicht stützen. Allgemein lässt sich also beobachten: Damit Wittgenstein in irgendeiner Form argumentieren kann, müssen seine Aussagen eine logische Struktur aufweisen.\footnote{Natürlich gibt es z.B. mit statistischen Verallgemeinerungen, Anwendungen solcher Verallgemeinerungen und Analogieargumenten noch viele weitere induktive Argumente, die ich hier nicht explizit betrachte. Da es aber plausibel ist, Analogieargumente als implizite Schlüsse auf die beste Erklärung zu verstehen, trifft meine Kritik zumindest dieses Verständnis. Die statistische Verallgemeinerung betrachte ich nicht, weil Wittgenstein keine einzelnen empirischen Daten zusammenträgt, sondern sich von Anfang an allgemein ausdrückt. Auf dieselbe Weise betreibt Wittgenstein auch keine statistische Anwendung in dem Sinn, dass der TLP Verhältnisse zwischen Mengen feststellt und aufgrund von diesen mit einer gewissen Wahrscheinlichkeit darauf schließt, dass einzelne Individuen in die eine oder andere Menge gehören. Selbst wenn er das täte, ließe sich aber auf ähnliche Weise zeigen, dass das eine logische Struktur seiner Aussagen voraussetzen würde. Für eine Begründung dafür, dass sich Analogieargumente als Schlüsse auf die beste Erklärung verstehen lassen Vgl. \cite{tetens2015}, Kapitel 15: Analogieargumente und \cite{sinnott-armstrong2014}, \emph{Are Analogies Explanations?}. Für eine allgemeine Erklärung der verschiedenen Arten von induktiven Argumenten vgl. \cite{sinnott-armstrong2014}, \emph{Part III: How to Evaluate Arguments: Inductive Standards}.} Das können sie aber nicht, wenn sie unsinnig sind. Wittgenstein kann also seinen eigenen Ansprüchen nach keine Gründe für seine Position formulieren und daher sollten wir sie zurückweisen. Nennen wir dieses Problem \textbf{das Problem der argumentativen Struktur des TLP}. Insgesamt stehen wir damit vor vier Problemen:

\begin{enumerate}
  \item Dem Problem der Selbstwiderlegung des Philosophie-Skeptizismus
  \item Dem Problem der Grenzen des Zeigbaren
  \item Dem Problem der Kommunikation des TLP
  \item Dem Problem der argumentativen Struktur des TLP
\end{enumerate}

Diese Probleme werfen Zweifel daran auf, dass Wittgensteins TLP ein plausibles Verständnis für die Grenzen unserer Sprache bietet. Vielleicht entstehen sie aber nur aufgrund von meiner vorläufigen naiven Interpretation. Im Folgenden werde ich daher Hackers, Conants und Moyal-Sharrocks Interpretationen Wittgensteins rekonstruieren und diskutieren, inwiefern sie mit diesen Problemen umgehen können.

% TODO: Meine Probleme resultieren daraus, dass ich Wittgenstein mit dem Anspruch lese die Grenzen für Sprache schlechthin zu formulieren, während er dazu nur Sprache gebraucht. Zeigen als Metasprache zu verstehen schwächt diesen Anspruch ab.

\section{Wie sollen wir diesen Unsinn verstehen?}

% Ich hab jetzt ne substanzielle Lesart vorausgesetzt. Also zuerst Conants Kritik rekonstruieren. Dann Hackers Replik und zum Schluss vtll noch mehr Hacker.

% Vielleicht sollte ich das ganze noch von der positivistischen Interpretation abgrenzen?

Die Leiter-Probleme resultieren daraus, dass Wittgensteins Verständnis von Unsinn und der Art und Weise, auf die Unsinn kommunizieren kann, auf den ersten Blick selbstwiderlegend wirkt. Um diese Probleme zu klären, entstand die erste Interpretation, die ich hier betrachten will, die (im Nachhinein so genannte) substanzielle Lesart. Im folgenden Abschnitt werde ich ihren prominentesten Vertreter Peter Hacker diskutieren. Die substanzielle Lesart ist aber noch in einigen Hinsichten unbefriedigend. Im Zuge einer alternativen Auseinandersetzung mit Wittgensteins Beziehung zu Frege und aufgrund einer Unzufriedenheit mit der substanziellen Lesart, entstand daher die vor allem von Cora Diamond und James Conant geprägte resolute Lesart. Im nächsten Kapitel werde ich daher Conants Auffassung rekonstruieren. Zum Schluss gehe ich noch auf Danièle Moyal-Sharrocks Vermittlungsversuch zwischen diesen Positionen ein. Es wird sich leider zeigen, dass keiner der Ansätze alle Leiter-Probleme befriedigend löst. Ich werde also zeigen, dass Wittgensteins TLP keine plausible Art bietet, die Grenzen unserer Sprache zu verstehen.

\subsection{Hackers substanzielle Lesart}

% Alternative Interpretation: Wittgenstein vermittelt uns know-how

% TODO: Wittgensteins Zeug zu logischer Syntax hab ich im Grunde oben schon erklärt. Etwas redundant hier.

Peter Hacker betont, dass Wittgenstein die philosophischen Probleme lösen will, indem er aufzeigt, dass sie auf Missverständnissen der Logik unserer Sprache beruhen.\autocite[Vgl. im Folgenden][I.3. Philosophy and illusion]{hacker1972} Wittgenstein unterscheidet nach ihm dabei zwischen Oberflächen- und Tiefengrammatik unserer Sprache. Ein Satz wie \enquote{Die Welt zerfällt in Tatsachen.} mag auf den ersten Blick wohlgeformt aussehen. Anders als \enquote{Die zerfallender Tatsachen, dass Welt.} weist er zum Beispiel keine offensichtlichen syntaktischen Fehler auf. Nach Hacker nimmt Wittgenstein aber an, dass unserer alltäglichen Syntax eine logische Syntax zugrunde liegt und dass die oben geschilderten Sätze den Regeln dieser nicht gehorchen. Diese logische Syntax ist es, die Wittgenstein durch seine Rede von Namen, Bildern und Sachverhalten erläutern will. Damit wir das können, brauchen wir mit \enquote{Name}, \enquote{Sachverhalt} oder \enquote{Gegenstand} aber formale Begriffe. Diese sind nicht wirklich Begriffe unserer Sprache, sondern unsinnige, scheiternde Versuche, sich auf Aspekte von Wittgensteins idealer formaler Notation zu beziehen, die sich in dieser einfach zeigen. Dass \enquote{$a$} und \enquote{$b$} Namen sind und der Sachverhalt behauptet wird, dass sie auf $R$ zusammenhängen, zeigt sich zum Beispiel in dem Satz \enquote{$aRb$}. Sagen können wir das streng genommen nicht. Formale Begriffe werden daher nur benötigt, um die Idealsprache das erste Mal zu erläutern. Ist diese einmal verstanden, braucht man sie nicht mehr, man kann die Leiter umstoßen.

Hacker und ich sind uns also einig darin, dass Wittgenstein mit seiner Forderung nach einer Idealsprache in die Fußstapfen von Leibniz, Frege und Russell tritt. Ein wichtiger Unterschied zwischen Wittgenstein und Russell bleibt aber, dass Wittgenstein zufolge jede mögliche Sprache ideal ist. Russell nimmt an, dass die Alltagssprache mangelhaft ist, weil sie es erlaubt, Vages und Ambiges zu sagen. Das geht bei Wittgenstein nicht. Wir können nur überhaupt etwas sagen, insofern wir den Regeln der logischen Syntax gehorchen. Damit gibt Wittgensteins Idealsprache die Struktur von Sprache schlechthin vor.\autocite[Vgl.][13]{hacker1972}

Sinnvolle Sätze wie die, die wir im Alltag oder in den Naturwissenschaften äußern, gehorchen demnach den Regeln der logischen Syntax. Sie sagen etwas über die Welt. Die Sätze der Logik widersprechen den Regeln der Syntax zwar nicht, sind aber trotzdem eine degenerierte Form von Satz, die nichts über die Welt aussagt, sondern nur ihre Struktur zeigt. Nur unsinnige Sätze widersprechen den Regeln der logischen Syntax: Sie sagen nichts und zeigen nichts. Gute Philosophie löst ihre Probleme Wittgenstein zufolge dann, indem sie aufzeigt, dass diese unsinnig sind.

Wittgenstein schildert eine streng korrekte Methode für die Philosophie.\autocite[Vgl. zu Wittgensteins philosophischer Methode][Kapitel I.2]{hacker1972} Eine PhilosophIn soll keinen Unsinn reden. Stattdessen soll sie Sätze, die die Welt abbilden, klären, indem sie mithilfe der formalen Notation ihre logische Struktur aufzeigt. Dabei muss sie den ein oder anderen Satz analysieren, also in Sätze zerlegen, aus denen er durch Wahrheitsfunktionen zusammengesetzt ist. Wenn nötig bis zu den Elementarsätzen. Das heißt, sie muss Sätze, die in philosophischen Debatten geäußert werden, in Wittgensteins Idealsprache übersetzen. Bei manchen Sätzen wird diese Übersetzung nicht gelingen. Hier zeigt die PhilosophIn durch ihre Klärungsversuche, dass die Sätze Unsinn sind.

Russell und Frege verstoßen gegen diese streng korrekte Methode. Indem sie ihre Idealsprachen erklären, versuchen sie zu sagen, was nur gezeigt werden kann. Denn sie verwenden formale Begriffe wie \enquote{Gegenstand} oder \enquote{Funktion}. Diese formalen Begriffe kommen in der Idealsprache nicht mehr vor. Anstatt zu sagen, dass etwas ein Gegenstand ist, verwenden wir ein Zeichen wie \enquote{$x$}, in einem Satz wie \enquote{$\exists x.fx$} und dass es ein Gegenstand ist, zeigt sich einfach.

Doch nicht einmal Wittgenstein selbst hält sich an seine streng korrekte Methode. Seine Sätze sind unsinnig. Damit stoßen wir auf das erste Leiter-Problem: Wittgensteins Philosophieskeptizismus scheint selbstwiderlegend zu sein.

Hacker korrigiert diesen Eindruck, indem er zwischen verschiedenen Arten von Unsinn unterscheidet.\autocite[Vgl.][18]{hacker1972} Manche Sätze sind \emph{offen} unsinnig, während andere \emph{verdeckt} unsinnig sind. Ein berühmtes Beispiel für offenen Unsinn ist Noam Chomskys \enquote{Farblose grüne Ideen schlafen wütend.} Wittgenstein bringt als Beispiel die Frage: \enquote{Ist gut mehr oder weniger identisch als schön?}.

Vor allem unterscheidet Hacker aber zwischen \emph{irreführendem} und \emph{einleuchtendem} Unsinn. Wittgenstein will mit dem TLP zeigen, dass die meisten philosophischen Fragen und Antworten in irreführendem verdecktem Unsinn bestehen. Er selbst schreibt einleuchtenden Unsinn. Dieser schafft dabei zwei Dinge gleichzeitig. Erstens hilft er uns, zu erfassen was andere Sätze zeigen, und zweitens zeigt er auf, dass er selbst zurückzuweisen ist. Irreführender Unsinn verwirrt uns dagegen einfach nur.\autocite[Vgl.][18]{hacker1972}

So löst Hacker das \textbf{Problem der Selbstwiderlegung des Philosophie-Skeptizismus} also, indem er bestreitet, dass philosophische Positionen automatisch dadurch widerlegt werden, dass sie als unsinnig entlarvt werden. Es kann sich schließlich immer noch um einleuchtend unsinnige Positionen handeln.

Damit beißt sich die Schlange zwar nicht mehr selbst in den Schwanz, aber nur, weil Hacker ihr die Zähne gezogen hat. Denn im Sinne einer Argumentationslinie Conants lässt sich bemerken, dass eine PhilosophIn, die von Wittgenstein durch den Nachweis der Unsinnigkeit ihrer Position kritisiert wird, einfach behaupten könnte, dass auch ihre Position einleuchtend unsinnig sei.\autocite[Conant bringt dieses Argument nicht explizit für Wittgenstein, sondern während einer Diskussion von Rudolf Carnaps Ansatz, Grenzen der Sprache zu ziehen, durch die Carnap versucht, Martin Heidegger zu kritisieren. Heidegger kann einfach selbst den Anspruch haben, eine Art von einleuchtendem Unsinn zu artikulieren. Vgl.][409]{conant2002} Um so eine PhilosophIn immer noch kritisieren zu können, braucht Wittgenstein eindeutige Kriterien dafür, welcher Unsinn einleuchtend und welcher irreführend ist. Das muss sich dann wohl an formalen Begriffen entscheiden. Unsinn ist genau dann einleuchtend, wenn er formale Begriffe verwendet, die scheiternd versuchen, auf unaussprechliche Aspekte der Idealsprache Bezug zu nehmen. Damit ist Russells und Freges Unsinn dann auch nicht mehr grundlegend fehlgeleitet, ihre Probleme sind nur, dass sie in manchen Details der Idealsprache falsch liegen.

Mit dieser Erklärung können wir das \textbf{Problem der Selbstwiderlegung des Philosophie-Skeptizismus} vorerst auf Distanz halten. Kaum klammern wir uns an den Sprossen der Leiter fest, stößt sie Wittgenstein aber mit dem Problem der \textbf{Grenzen des Zeigbaren} wieder um. Denn Wittgenstein gibt keine klaren Kriterien an, anhand derer sich ermitteln lässt, ob ein Pseudobegriff in einem unsinnigen Satz als formaler Begriff in Frage kommt. Schließlich können wir über so etwas nicht reden, es soll sich einfach zeigen. Indem Wittgenstein auf diese Weise den Inhalt seiner Philosophie in die diffuse Kommunikationsform des Zeigens verschiebt, beantwortet er die Frage nach den Grenzen der Sprache aber nicht und öffnet der Metaphysik wieder die Tore. Denn jetzt kann jede MetaphysikerIn, die eine neue Ordnung der Welt als solcher aufstellen will, sich einfach darauf berufen, dass ihre Worte zwar unsinnig sind, sie aber formale Begriffe verwendet, deren pseudo-Geltung sich von ganz alleine zeigt, wenn man nur gewöhnliche Sätze betrachtet.

Hacker selbst lässt Wittgenstein auf diese Probleme nicht mehr antworten, weil er glaubt, dass Wittgenstein sie nicht beantworten kann. Er formuliert meine Sorgen folgendermaßen:

\begin{quote}
  The doctrine [of the inexpressibility of what is shown in a conceptual notation] rests on a variety of premises which are assumed rather than properly argued for in the \emph{Tractatus}, all of which were later repudiated by Wittgenstein. We are given no clue, in the \emph{Tractatus}, how to determine which concepts are formal concepts and which are not.\footnote{\cite{hacker1972}, S. 23. \textbf{Deutsch:} Die Lehre der Unausdrückbarkeit von dem, das sich in einer begrifflichen Notation zeigt, beruht auf einer Vielzahl an Prämissen, die im \emph{Tractatus} angenommen und nicht anständig begründet werden und die Wittgenstein später alle zurückweist. Wir bekommen im \emph{Tractatus} keinen Hinweis darauf, wie sich herausfinden lässt, welche Begriffe formal sind und welche nicht.}
\end{quote}

Einleuchtender Unsinn ließe sich nur von irreführendem Unsinn unterscheiden, wenn wir feststellen könnten, in welchem formale Begriffe vorkommen und in welchem nicht. Indem Hacker kritisiert, dass Wittgenstein keine Kriterien dafür angibt, welche Begriffe formal sind, merkt er also, dass der TLP keine Grenzen dessen ziehen kann, was gezeigt werden kann und damit auch, was durch einleuchtenden Unsinn kommuniziert werden kann. Selbst Hacker weigert sich also, Wittgensteins wackelige Leiter zu erklimmen.

Aber auch wenn wir diese Probleme außen vor lassen, disktuiert Hacker die Probleme der \textbf{Kommunikation des TLP} und der \textbf{Argumentativen Struktur des TLP} nicht einmal. Denn der TLP ist ein Versuch, zu erklären, wie wir durch sprachliche Äußerungen kommunizieren. Die Grenzen unserer Sprache sollen diese Kommunikationsfähigkeiten abtasten und ihre Einschränkungen aufzeigen. Wenn Wittgenstein mit dem TLP jetzt aber mit Sagen und Zeigen zwei Formen des sprachlichen Ausdrucks annimmt, von denen er nur eine erklärt und, wenn er für seine eigenen Äußerungen annehmen muss, dass diese unter eine dritte, überhaupt nicht erklärte Form der Kommunikation durch einleuchtenden Unsinn fallen, dann ist dieser Versuch gerade nicht gelungen. Denn all die Ausführungen zu Sachlagen und Elementarsätzen helfen uns, um einleuchtenden Unsinn zu verstehen, nicht weiter. Wie und warum wir ihn verstehen können, bleibt rätselhaft. Rätselhaft bleibt auch, wie der TLP Argumente enthalten kann, wenn er keine logische Struktur hat. Er besteht aus Sätzen, in denen nur Pseudobegriffe vorkommen, vielleicht hat er also auch nur eine pseudoargumentative Struktur? Und wenn wir so nett sind, diese Pseudoargumente, die uns keine Wahrheit garantieren, zu akzeptieren, verspricht uns Wittgenstein tiefgreifende Erkenntnisse?

Fassen wir all das noch einmal zusammen. Es gelingt Hacker, auf das \textbf{Problem der Selbstwiderlegung des Philosophie-Skeptizismus} zu antworten. Denn indem er \enquote{einleuchtendem Unsinn} zulässt, schwächt Hacker den Philosophie-Skeptizismus so stark ab, dass er Wittgenstein nicht mehr trifft. Dafür trifft er aber auch niemand anderen mehr. Denn das \textbf{Problem der Grenzen des Zeigbaren} bleibt bestehen und das gesteht Hacker sogar zu. Auf die Probleme der \textbf{Kommunikation des TLP} und der \textbf{Argumentativen Struktur des TLP} geht Hacker nicht mehr ein. Auch sie bleiben aber unter seiner Interpretation schwerwiegende Einwände gegen Wittgensteins Projekt.

\subsection{Conants resolute Lesart}

% Ein Vorteil: Inhalt des TLP hängt untrennbar mit seiner Form zsm. Wenn ich den TLP als eine Art, eine allgemeine Theorie zu verteidigen sehe, muss ich das fallen lassen.

% Conant nennt das, das ich hier "substanzielle Lesart" nenne, "Un\-aus\-sprech\-bar\-keits-Lesart". Hacker hat auch andere Namen

Als nächstes werde ich mit James Conant die Antworten auf die Leiter-Probleme von einem der paradigmatischen Vertreter der resoluten\footnote{Conant nennt seine eigene Position nicht \enquote{resolut}, sondern \enquote{austere}\footnote{Deutsch: streng, asketisch, karg, schmucklos.}. Da \enquote{resolut} sich aber als Kategorie herausgestellt hat, durch die Lesarten wie seine in der Literatur allgemein bezeichnet werden, weiche ich hier von seiner Terminologie ab.} Lesart des TLP diskutieren.\footnote{James Conant und Cora Diamond sind die klassischen VertreterInnen der resoluten Lesart. Ich diskutiere hier Conant, weil sein klassischer Text \emph{The Method of the Tractatus} klarer und zugänglicher ist als Diamonds \emph{The Realistic Spirit}. Vgl. daher im Folgenden \cite{conant2002}. Für Diamond vlg. \cite{diamond1995}.} Um zu verstehen, was Conant positiv zur Debatte beiträgt, ist es hier zunächst wichtig, zu verstehen, wie er Hacker kritisiert. Denn polemisch ausgedrückt ist nach Conant der TLP nichts weiter als eine \emph{Parodie} der substanziellen Lesart. Wittgenstein versucht nicht, einleuchtenden Unsinn zu artikulieren, der unaussprechliche Wahrheiten vermittelt. Stattdessen äfft er diesen Versuch auf eine Weise nach, die uns klar machen soll, dass dabei einfach nur Unsinn herauskommt. Ganz unabhängig von der Frage danach, ob Conants Kritik an Hacker stichhaltig ist, brauchen wir sie also, um zu begreifen, was genau der TLP Conant zufolge parodiert. Und es ist wichtig, das zu verstehen, um genauer zu wissen, wie Wittgensteins philosophische Kritik nach Conant funktioniert. Conant stützt seine Interpretation erstens auf Wittgensteins Kontextprinzip und seine Unterscheidung zwischen Zeichen und Symbol. Zweitens bietet er eine Fehlertheorie an: Die Position, die substanzielle LeserInnen Wittgenstein unterstellen, geht eigentlich auf Frege zurück, welchen Wittgenstein kritisiert. Kommen damit zu Conants Kritik.

Meine Leiter-Probleme resultieren daraus, dass Wittgensteins TLP sich selbst zufolge unsinnig ist: Wenn er unsinnig ist, wieso widerlegt er sich nicht selbst, wie kann er dann noch Grenzen der Sprache ziehen, wie kommuniziert er und wie kann er Argumente enthalten? Dabei sind zwei Begriffe besonders interessant: \emph{Unsinn} und \emph{Erläuterung}. Resolute LeserInnen wie Conant und substanzielle Leser wie Hacker streiten sich hauptsächlich darum, wie diese Begriffe bei Wittgenstein zu verstehen sind.

Diese Begriffe sind so wichtig, weil Wittgenstein im \textbf{Leiter-Satz} schreibt, dass seine Worte dadurch \enquote{\emph{erläutern}}, dass sie \enquote{der, welcher [ihn] versteht, am Ende als \emph{unsinnig} erkennt}.\footnote{Vgl. \cite{wittgenstein1922}, Vorwort, meine Markierung.} Auch, dass Wittgenstein Carnap vorwarf \enquote{die grundlegende Konzeption des Buchs}\footnote{Meine Übersetzung von \enquote{the fundamental conception of the book} zitiert nach \cite{conant2002}, S. 378, oben.} misszuverstehen, weil dieser den \textbf{Leiter-Satz} falsch interpretiere, betont dessen Wichtigkeit. Dazu kommt, dass Wittgenstein es in einem Brief an den ersten Übersetzer des TLP, C. K. Odgen, ablehnte, ergänzendes Material zu drucken, weil dieses keine Erläuterungen enthalte.\footnote{Vgl. \cite{wittgenstein1973}, S. 46. Zitiert nach \cite{conant2002}, S. 378.} Außerdem veränderte Wittgenstein in Odgens Übersetzung des \textbf{Leiter-Satzes} \enquote{explain} (erklären) zu \enquote{elucidate} (erläutern).\footnote{Vgl. \cite{wittgenstein1973}, S. 51 und S. 53-54. Zitiert nach \cite{conant2002}, 379.} All das zeigt, dass Wittgenstein besonderen Wert darauf gelegt hat, zu betonen, dass der TLP in Erläuterungen besteht und dass diese Erläuterungen erläutern, indem sie als Unsinn erkannt werden.

% Conants Kriterien?: Keine Lehre + Erläuterung. Nochmal Conant lesen

Um auf die Leiter-Probleme zurückzukommen, würde Conant also auf \textbf{das Problem der Kommunikation des TLP} und auf \textbf{das Problem der argumentativen Struktur des TLP} antworten, dass der TLP nicht in gewöhnlichen Aussagen oder Argumenten, sondern in Erläuterungen besteht. Diese Erläuterungen sollen erläutern, indem sie die, die Wittgenstein verstehen, als unsinnig erkennen. Das wirft aber nur die neue Frage auf, was genau mit diesem Verständnis von Erläuterung gemeint ist und was es hier heißt, unsinnig zu sein.

Beginnen wir mit Unsinn. Conant unterstellt Hacker eine substanzielle Un\-aus\-sprech\-bar\-keits-Lesart von Unsinn, welche er von positivistischen Lesarten abgrenzt.\footnote{In Fußnote 2 nennt Conant als paradigmatische Vertreter der Un\-aus\-sprech\-bar\-keits-Lesart Peter Hacker und Peter Geach. Unter positivistischen Lesern versteht er nach Fußnote 1 in erster Linie den Wiener Kreis.} Nach Conant nehmen viele Un\-aus\-sprech\-bar\-keits-Lesarten an, dass einleuchtender Unsinn zeigt, was nicht gesagt werden kann.\autocite[][376]{conant2002} Ich habe oben schon gegen diese Auffassung argumentiert und sowohl Conant als auch Hacker teilen diese Kritik.\footnote{Vgl. \cite{conant2002}, Fußnote 4 und \cite{hacker1972}, S. 18.} In einer Fußnote räumt Conant auch ein, dass diese Kritik Hacker nicht trifft.\autocite[Vgl.][Fußnote 69]{conant2002} Aufgrund der Art und Weise, auf die Hacker Wittgensteins Kommunikation durch Unsinn versteht, ordnet Conant ihn aber trotzdem als substanziellen Leser ein. Dazu später mehr.

Conant schildert weiter, dass Un\-aus\-sprech\-bar\-keits-LeserInnen annehmen, dass Wittgensteins Unsinn absichtlich die Regeln logischer Syntax verletzt, um Gedanken zu kommunizieren, die sich nicht sagen lassen. Diese Gedanken beziehen sich auf Merkmale der Wirklichkeit (\emph{features of reality}), auf die wir uns mit wohlgeformten Sätzen nicht beziehen können. So unterscheiden Un\-aus\-sprech\-bar\-keits-LeserInnen zwischen zwei Arten von Gedanken: Solchen innerhalb der Grenzen der Logik und solchen außerhalb.\autocite[Vgl.][376]{conant2002} Hier fragt Conant: Um was für eine Art von Unsinn handelt es sich? Was für ein Gedanke wird kommuniziert? Und wie kann das trotz eines Verstoßes gegen die Regeln logischer Syntax gelingen?\autocite[Vgl.][Fußnote 5]{conant2002} Conant wirft also in anderen Worten \textbf{das Problem der Kommunikation des TLP} für substanzielle Un\-aus\-sprech\-bar\-keits-Lesarten auf. Er wird letztendlich dafür argumentieren, dass sich dieses Problem nur im Rahmen einer substanziellen Lesart stellt und dass er es mit seiner resoluten Lesart lösen kann.

% TODO: Das gleich durch Zeichen/Symbol charakterisieren!

Um Wittgenstein richtig zu interpretieren, müssen wir, so Conant, erkennen dass er auf Freges Kontextprinzip aufbaut. Das Kontextprinzip ist eins der drei Prinzipien, die Frege zu Beginn seiner \emph{Grundlagen der Arithmetik} festlegt:

\begin{enumerate}
  \item es ist das Psychologische von dem Logischen, das Subjektive von dem Objektiven scharf zu trennen;
  \item \textbf{Kontextprinzip:} nach der Bedeutung der Wörter muss im Satzzusammenhange, nicht in ihrer Vereinzelung gefragt werden;
  \item der Unterschied zwischen Begriff und Gegenstand ist im Auge zu behalten.\autocite[][]{frege1884}
\end{enumerate}

Diese Prinzipien sind eng verknüpft. Wir missachten die Trennung zwischen logischem und psychologischem zum Beispiel dadurch, dass wir die Bedeutungen von Wörtern durch psychische Zustände wie Bilder vor dem inneren Auge erklären. Denn das geht nur, indem wir nach der Bedeutung isolierter Worte fragen, ohne sie im Kontext des Satzes zu betrachten. Auch die Trennung zwischen Begriff und Gegenstand zeigt sich nur im Kontext eines Satzes. Conant zufolge setzt Wittgenstein diese Prinzipien voraus, entwickelt sie aber weiter und modifiziert sie.\autocite[Vgl.][384]{conant2002}

Wittgenstein wiederholt das Kontextprinzip in eigenen Worten als \enquote{Nur der Satz hat Sinn; nur im Zusammenhange des Satzes hat ein Name Bedeutung.}\footnote{\cite{wittgenstein1922}, 3.3.} Darauf folgt im Text seine Unterscheidung zwischen \emph{Zeichen} und \emph{Symbol}.\footnote{Für Wittgensteins Diskussion der Unterscheidung zwischen Zeichen und Symbol vgl. \cite{wittgenstein1922}, 3.3-3.328. Conant diskutiert diese Unterscheidung in \cite{conant2002}, Abschnitt IX - The Tractarian Critique of the Substantial Position.} Symbole sind die Teile eines Satzes, die für seinen Sinn relevant sind. Zeichen sind dagegen nur das sinnlich wahrnehmbare am Symbol. Der Unterschied lässt sich an folgenden Sätzen veranschaulichen:

\begin{itemize}
  \item Bruce Wayne ist Batman.
  \item Bruce Wayne ist Multimilliardär.
  \item Bruce Wayne ist.
\end{itemize}

In all diesen Sätzen kommt dasselbe sinnlich wahrnehmbare Zeichen \enquote{ist} vor. Es symbolisiert aber immer auf verschiedene Weise. Wittgenstein würde sagen, dass es jeweils als Kopula, Gleichheitszeichen und Existenzquantor funktioniert. Er fordert daher, dass wir solche Sätze in seiner formalen Idealsprache darstellen, um die Unterschiede zu zeigen. Hier sind sie beispielsweise in zeitgenössischer Logik:

\begin{itemize}
  \item $w = b$
  \item $M(w)$
  \item $\exists x.W(x)$
\end{itemize}

% TODO: Beziehung zu drittem Prinzip? ==> Kontextprinzip im Zentrum

Das substanzielle Verständnis von Unsinn, das Conant Hacker unterstellt, unterscheidet zwischen \emph{substanziellem-} und \emph{einfach-nur-Unsinn}. Substanzieller Unsinn besteht aus einzeln intelligiblen Komponenten, die aber auf eine Weise kombiniert werden, die die logische Syntax verbietet. Sowohl \enquote{Sokrates ist identisch.} als auch \enquote{Die Welt zerfällt in Tatsachen.} sind demnach solcher substanzieller Unsinn. Nach Conant unterscheiden sich die Un\-aus\-sprech\-bar\-keits- und die positivistische Lesart darin, \enquote{wo [diese] Verletzung logischer Syntax stattfindet}.\footnote{Original: \enquote{where the violation transpires when a transgression of logic occurs}, \cite{conant2002}, S. 392. Das Zitat stammt von einer späteren Stelle im Text, an der er den Gedanken zusammenfasst. Zum ersten Mal beschreibt er ihn auf S. 381.} Der positivistischen Lesart zufolge verletzen \emph{die unsinnigen Sätze} die Regeln der logischen Syntax, nach der Un\-aus\-sprech\-bar\-keits\-les\-art können das nur \emph{die unaussprechbaren Gedanken}, die von diesen gezeigt werden. Einfach-nur-Unsinn dagegen hat überhaupt keine intelligiblen Komponenten. Ein Beispiel für so einen Satz wäre: \enquote{Wimpf Düdelklopps klimp-tutteln?}

Dieses substanzielle Verständnis von Unsinn grenzt Conant von seinem \emph{resoluten}\footnote{Conant nennt sein Verständnis von Unsinn und Erläuterung, wie auch seine Position eigentlich \enquote{austere}. Ich lege ihm hier \enquote{resolut} in den Mund, um die Terminologie nicht noch weiter zu verkomplizieren.} Verständnis ab. Dem zufolge gibt es ausschließlich \emph{einfach-nur-Unsinn}.

Letztendlich diskutiert Conant diese unterschiedlichen Verständnisse von Unsinn mithilfe von Wittgensteins Unterscheidung zwischen Zeichen und Symbol. Nach der substanziellen Konzeption gibt es substanziellen Unsinn, der in einer \enquote{Kollision} (clash) der logischen Kategorien von Symbolen besteht. Nach der resoluten Lesart gibt es dagegen ausschließlich einfach-nur-Unsinn, also Zeichen ganz ohne symbolische Funktion. Nachdem er die substanzielle Lesart auf diese Weise formuliert hat, stellt Conant die positivistische Lesart vor ein Dilemma: Wenn sie annimmt, dass die logische Syntax nur Kombinationen von Zeichen verbietet, wird sie zur resoluten Lesart. Denn dann werden die Zeichen nie zu Symbolen und damit ist auch keine logische Struktur im Unsinn erkennbar. Unsinn ist dann immer einfach-nur-Unsinn. Wenn die positivistische Lesart aber daran festhält, das substanzieller Unsinn in einer unerlaubten Verwendung von Symbolen besteht, fällt sie in die Un\-aus\-sprech\-bar\-keits-Lesart zurück. Denn dann symbolisieren die Worte etwas, dessen Kombination verboten ist, wofür wir uns diese Kombination zumindest denken können müssen.\autocite[Für Conants Diskussion der Lesarten mithilfe von Wittgensteins Unterscheidung zwischen Zeichen und Symbol vgl.][400-401]{conant2002}

Mit diesen Verständnissen von Unsinn gehen auch unterschiedliche Verständnisse von Erläuterung einher. Nach substanziellem Verständnis soll eine Erläuterung etwas zeigen, das nicht gesagt werden kann. Sie besteht also, auf die Un\-aus\-sprech\-bar\-keits-Lesart bezogen, in der Kommunikation genau des Gedankens, der nach den Regeln der logischen Syntax verboten ist. Nach resolutem Verständnis dienen Erläuterungen hingegen nur dazu, aufzuzeigen, dass wir uns bei bestimmten Sätzen bloß eingebildet hatten, etwas mit ihnen zu meinen, dass wir aber eigentlich nichts mit ihnen meinten.\autocite[Für die beiden Verständnisse von Unsinn und Erläuterung vgl.][380-381]{conant2002}

Dass Hacker ein substanzielles Verständnis von Erläuterung vertritt, belegt Conant dadurch, dass Hacker schreibt, dass Wittgenstein mit seinem Unsinn \enquote{etwas meinte}, welches er für \enquote{korrekt} hielt und von dem er glaubte, dass es von anderen \enquote{verstanden} (grasped) werden könnte.\footnote{Vgl. \cite{hacker1972}, S. 18-19, 26. Zitiert nach \cite{conant2002}, S. 393.} All diese Arten zu reden setzen aber voraus, dass Wittgenstein glaubte, einen gewissen Inhalt zu kommunizieren. Diesen Inhalt müssen wir uns wohl als so etwas wie einen logisch verbotenen Satz vorstellen, der sich zwar nicht sagen, aber durch Unsinn trotzdem andeuten lässt.% TODO: Kann man dasselbe Spiel nicht auch mit Conant spielen?

% TODO: Wie ist das mit der Vorstellung von Unsinn

Diese Interpretation scheitert, wie wir es auch oben schon bemerkt haben, an dem \textbf{Problem der Kommunikation des TLP}: Wenn die Gedanken wirklich unaussprechlich und logisch verboten sind, wieso kann Wittgenstein sie dann mithilfe sprachlich artikulierten Unsinns kommunizieren? Doch Conant sieht das nicht als Kritik an Wittgenstein, sondern zieht daraus den Schluss, dass Hackers substanzielle Lesart nur eine \enquote{dialektische Vorstufe} zu Wittgensteins Position ist.\autocite[][377]{conant2002} Ihm zufolge kämpft schon Freges Philosophie mit der Spannung zwischen substanziellen und resoluten Konzeptionen von Unsinn und Erläuterung. Nach Conant erkennt Wittgenstein diese Spannungen und löst sie zugunsten der resoluten Konzeption auf. Hacker missverstehe Wittgenstein, weil er den Teil von Frege in Wittgenstein hineinlese, \emph{gegen} den dieser eigentlich argumentiere.\autocite[][381]{conant2002}

Die Spannung, die Wittgenstein nach Conant bei Frege löst besteht zwischen der Idee, dass sich die Begriffsschrift nicht im Rahmen der Begriffsschrift erläutern lässt, auf der einen Seite und dem Kontextprinzip auf der anderen Seite. Frege ist sich der Grenzen seiner Begriffsschrift bewusst. Er ist darauf angewiesen, dass seine LeserInnen einen Gedanken, der streng genommen nicht in die klare logische Form passt, trotzdem akzeptieren. Für Conant zeigt das, dass Frege zu einem substanziellen Verständnis von Unsinn gedrängt ist.\footnote{Vgl. \cite{conant2002}, \enquote{III - The Neglect of Frege?}, \enquote{IV - Frege on Concept and Object}, \enquote{V - Fregean Elucidation}, \enquote{VI - Elucidatory Nonsense} und insbesondere das Frege-Zitat auf S. 390.} Wie ich oben schon erklärt habe, lässt sich diese Konsequenz aber vermeiden, wenn man die Begriffsschrift als ein eingeschränktes Mittel für einen bestimmten Zweck versteht.

Entgegen dieser Positionen zwingt sein Kontextprinzip Frege, so Conant, zu einem resoluten Verständnis von Unsinn und Erläuterung. \footnote{Man könnte hier Freges berühmte Unterscheidung zwischen Sinn und Bedeutung heranziehen und sich fragen, ob gemeint ist, dass Satzkomponenten nur im Kontext des Satzes einen \emph{Sinn} oder eine \emph{Bedeutung} haben. Conant bemerkt hier, dass Frege sein Prinzip formulierte, bevor er begann, klar zwischen diesen Phänomenen zu unterscheiden. Für Conants Argument spielt das aber aufgrund der Art und Weise, wie Conant Wittgensteins Auflösung der Spannung versteht, keine Rolle. Denn nach Conant erkennt Wittgenstein, dass Sinn sich nicht von Bedeutung trennen lässt. Demnach haben Satzkomponenten nur im Kontext eines sinnvollen Satzes Bedeutung und ein Satz hat nur Sinn, wenn seine Komponenten Bedeutung haben. Vgl. \cite{conant2002}, Fußnote 32. Conant zitiert 3.3 dafür, dass Satzkomponenten nur im sinnvollen Satz Bedeutung haben und 5.473, 5.4733 und 5.53 dafür, dass Sätze erst dann Sinn haben, wenn ihre Komponenten etwas bedeuten.} Dass das Kontextprinzip keinen substanziellen Unsinn zulässt, lässt sich am einfachsten an einem von Wittgensteins Beispielen für Unsinn zeigen:

\claim{Wittgenstein, Unsinn, Sokrates (WUS)} Sokrates ist identisch.\footnote{\cite{wittgenstein1922}, 5.473.}\label{WUS}

Substanziellen Lesarten zufolge ist \textbf{WUS} unsinnig. Denn indem wir \enquote{Sokrates}, \enquote{ist} und \enquote{identisch} als Symbole verstehen, können wir den Ausdrücken logische Kategorien (z.B. Name, Prädikat) zuordnen.\footnote{In Wittgensteins Idealsprache kommen keine Prädikate vor. Ich verwende logische Kategorien wie \enquote{Prädikat} nur, weil es, wie ich im Teil zur Wittgenstein Rekonstruktion erkläre, sonst unpraktikabel ist, konkrete Beispiele zu formulieren. Meine alltagssprachlichen Beispiele veranschaulichen also die grundlegenden Prinzipien, die letztendlich auf Elementarsätze angewendet werden sollen.} Die logische Syntax erlaubt es aber nicht, Ausdrücke dieser Kategorien auf diese Weise zu verbinden. \enquote{Sokrates} ist ein Name, \enquote{ist} verbindet einen Namen mit einem Prädikat, aber \enquote{identisch} ist kein Prädikat, sondern eine Relation, die Ausdrücke derselben logischen Kategorie verknüpft.

Diese Analyse setzt voraus, dass \enquote{Sokrates} in \textbf{WUS} als Name funktioniert, weil \enquote{Sokrates} in unserer Sprache gewöhnlich als Name funktioniert. Sehen wir das so, missachten wir aber Freges Kontextprinzip. Betrachten wir etwa das folgende Beispiel:

\claim{Precht Sokrates (PS)} Richard David Precht ist ein moderner Sokrates.\autocite[Conant zitiert Freges Beispiel: \enquote{Trist ist kein Wien.}][398]{conant2002}

Nach der substanziellen Analyse oben müsste \textbf{PS} Unsinn sein. \enquote{Sokrates} ist ein Name, an seiner Stelle wäre aber ein Prädikat gefordert. Doch der Satz ist kein Unsinn, denn in ihm funktioniert \enquote{Sokrates} als Prädikat. Das zeigt, dass sich die logische Rolle eines Ausdrucks nicht allgemein, sondern immer nur im Kontext eines bestimmten Satzes feststellen lässt. Die substanzielle Erklärung scheitert also. Eine andere Formulierung der Kritik wäre diese: Der substanziellen Lesart zufolge sind Sätze aufgrund ihrer Symbole unsinnig. Nach dem Kontextprinzip können wir aber nur feststellen, welche Symbole ein Satz enthält, wenn der ganze Satz sinnvoll ist. Der Fall, den die substanzielle Lesart erfordert, kann also nach dem Kontextprinzip nicht auftreten.\autocite[Vgl.][Abschnitt IX - The Tractarian Critique of the Substantial Position.]{conant2002}

Fassen wir Conants Kritik an Hackers Un\-aus\-sprech\-bar\-keits-Lesart noch einmal zusammen. Nach Conant teilt sich Hacker mit positivistischen Lesern eine substanzielle Konzeption von Unsinn und Erläuterung. Außerdem beobachtet Conant, dass Wittgenstein Freges Kontextprinzip aufnimmt. Das substanzielle Verständnis von Unsinn und Erläuterung steht aber in Konflikt mit dem Kontextprinzip. Daher ist auch die Un\-aus\-sprech\-bar\-keits-Lesart Hackers abzulehnen. So weit zum negativen Teil von Conants Interpretation. Als nächstes werde ich schildern, für welche Interpretation Conant argumentiert.

Conant zufolge soll Wittgensteins logische Syntax weder Kombinationen von Zeichen, noch Kombinationen von Symbolen verbieten.\autocite[Vgl.][414]{conant2002} Obwohl die logische Syntax nach ihm nichts verbietet, können wir nach Conant aber trotzdem verstehen, was mit \enquote{Verletzung der logischen Syntax} gemeint sein könnte: Sätze wie \textbf{WUS} sind Conant zufolge logische Kategorien übergreifende Äquivokationen.\autocite[Vgl.][415-418]{conant2002} Die Äquivokation ist ein ungültiges Argument, das auf den ersten Blick gültig erscheint, weil zwei gleiche Zeichen mit unterschiedlicher Bedeutung verwendet werden. Ein Beispiel meines ehemaligen Philosophietutors namens Alexander Belak ist:

\begin{itemize}
  \item \textbf{P1:} Alex is ne Flasche.\footnote{\enquote{Flasche} wird hier als verspielte Beleidigung verstanden, die jemanden als unfähig und schwächlich bezeichnet.}
  \item \textbf{P2:} Flaschen sind durchsichtig.
  \item \textbf{K:} Alex ist durchsichtig.
\end{itemize} % TODO: Table!

Das Argument ist entgegen des anfänglichen Anscheins ungültig, weil, es sich nur verstehen lässt, wenn \enquote{Flasche} gleichzeitig für zwei verschiedene Prädikate steht. Es müsste im ersten Satz wortwörtlich oder im zweiten als Beleidigung gemeint sein. Damit wären die jeweiligen Prämissen aber nicht mehr plausibel.

Dieses Verständnis von Äquivokation lässt sich nur metaphorisch auf Conants Interpretation übertragen. Schließlich nimmt dieser an, dass die Zeichen in unsinnigen Sätzen überhaupt keine Bedeutung haben und dass diese Sätze erst recht keine logische Struktur haben. Doch Conant kann zugestehen, dass die Zeichen das hätten, wenn wir sie interpretieren könnten. So wäre eine Interpretation von \textbf{WUS} denkbar, bei der \enquote{identisch} für eine normale Eigenschaft wie zum Beispiel Mut steht. In dieser Interpretation könnten wir verstehen, was mit \enquote{Sokrates ist} gemeint ist. Sie ist aber für \enquote{identisch} unplausibel. Wir könnten den Satz auch so interpretieren, dass \enquote{Sokrates ist} eigentlich so viel bedeutet wie \enquote{Sokrates und Platon sind}. Interpretieren wir den Satz so, können wir verstehen, was mit \enquote{identisch} gemeint ist. Diese Interpretation ist aber für \enquote{Sokrates ist} unplausibel. Diese beiden Arten von Interpretationen unterscheiden sich dadurch, dass wir den jeweiligen Phrasen verschiedene logische Kategorien zurodnen. Mit der Aussage, dass Verletzungen logischer Syntax Äquivokationen über verschiedene logische Kategorien hinweg sind, meint Conant also, dass wir solche Sätze nur verstehen können, wenn wir verschiedene Interpretationen für sie gleichzeitig voraussetzen, die den Satzkomponenten verschiedene logische Kategorien zuordnen. % TODO: Vergleich mit Duck-Rabbit: Man kann nicht nur das Hasenauge mit Entennase haben;
\autocite[Vgl.][418-420]{conant2002}

Damit kommen wir jetzt endlich zu Conants Verständnis von Erläuterung. Nach Conant erläutern die unsinnigen Sätze des TLP dadurch, dass sie eine bestimmte Art von Erfahrung in der LeserIn hervorbringen. Zuerst wird sie zu einer Theorie von Sprache und Welt verlockt, nach der die Sprache die Welt abbildet, dieses Verhältnis sich aber sprachlich nicht ausdrücken lässt. Die LeserIn wird also erst von der substanziellen Konzeption von Unsinn überzeugt, nach der es unaussprechbare Einsichten gibt. Hier erweckt der TLP den Anschein, dass er aus Argumenten besteht, die Positionen stützen. Dann wird der Unsinn, der die LeserIn zu dieser Konzeption verleitete, aber als logische Kategorien übergreifende Äquivokation aufgezeigt. Dabei wird die Illusion von Sinn, zu der sie verleitet wurde, zerbrochen: Mit dem Unsinn war überhaupt nichts gemeint. So wirft die LeserIn die Leiter um. Sie erkennt den gesamten TLP als Unsinn. Dennoch hat sie sich aber an einer Praxis beteiligt und dabei eine besondere Erfahrung gemacht, die sie verändert hat. Denn jetzt, da sie zuerst getäuscht und dann aufgeklärt wurde, wird sie nicht mehr in die erstere Täuschung zurückfallen.\autocite[Vgl.][421-424]{conant2002} Conant beschreibt diese besondere Erfahrung folgendermaßen:

\begin{quote}
  On this reading, first I grasp that there is something that \emph{must} be; then I see that it cannot be said; then I grasp that if it cannot be said it cannot bei thought (that the limits of language are the limits of thought); and then, finally, when I reach the top of the ladder, I grasp that there has been not \enquote{it} in my grasp all along (that that which I cannot think I cannot \enquote{grasp} either).\footnote{\textbf{Deutsch:} Nach dieser Lesart begreife ich zuerst, dass etwas sein \emph{muss}; dann sehe ich, dass es nicht gesagt werden kann; dann begreife ich, dass, wenn es nicht gesagt werden kann, es auch nicht gedacht werden kann (also dass die Grenzen der Sprache die Grenzen des Denkens sind); und dann, letztendlich, wenn ich die letzte Sprosse der Leiter erreiche, begreife ich, dass da nie ein \enquote{etwas} war, das ich begriffen habe (also dass ich das, das ich nicht denken kann, auch nicht begreifen kann). \cite{conant2002}, S. 422.}
\end{quote}

Ich habe diese Gedanken oben in der Formulierung zusammengefasst, dass Conant den TLP als \emph{Parodie} der substanziellen Lesart versteht. Denn diese Bezeichnung bringt noch klarer hervor, auf welche Weise sich die Kritik des resoluten Wittgensteins von philosophischer Kritik, wie wir sie gewöhnlich verstehen, unterscheidet. Schließlich gehen wir gewöhnlich davon aus, dass philosophische Kritik entweder in Hinweisen auf die Ungültigkeit von Argumenten oder in Argumenten besteht, deren Konklusionen Prämissen oder zumindest Konklusionen von anderen Argumenten negieren. Eine Parodie dagegen kritisiert, indem sie ihren Gegenstand imitiert, dabei aber Aspekte des Gegenstands überspitzt darstellt, um deren Absurdität aufzuzeigen. Um ein Beispiel zu nennen: Jan Böhmermanns Lied \emph{Menschen Leben Tanzen Welt} besteht aus Kalendersprüchen, Tweets, Zeilen aus deutschen Popsongs und Werbebotschaften. Böhmermann bemängelt, dass deutsche PopmusikerInnen gehaltlose Texte schreiben und sich zu sehr für ihren Profit interessieren. Das Lied kritisiert die Popmusik aber nicht in Form von derartigen Thesen, sondern indem es die bemängelten Eigenschaften selbst in offensichtlicher und überspitzter Form aufweist. Wittgensteins TLP funktioniert nach Conant weniger wie ein klassisches philosophisches Argument und eher wie so eine Parodie. Schließlich lehnt Wittgenstein nach Conant Argumentation als solche letztendlich ab. Er tut aber zu Beginn des TLP so, als wollte er philosophische Argumente formulieren, er \emph{äfft} die gewöhnliche Vorstellung philosophischer Argumentation nach. Zum Schluss sollen wir an dieser Ausführung aber selbst erkennen, dass der Versuch unsinnig ist. % TODO: Man muss den Rahmen schon verständlich lassen. Nur der Körper ist eine heimliche Parodie.

An dieser Stelle will ich kurz innehalten und noch einmal zusammenfassen, wie Conant auf die verschiedenen Leiter-Probleme eingehen könnte, um im nächsten Schritt meine Kritik an ihm zu formulieren. Beginnen wir mit den Problemen der \textbf{Grenzen des Zeigbaren} und der \textbf{Kommunikation des TLP}. Ich habe hier behauptet, dass Wittgenstein mit dem Zeigen eine zweite Ausdrucksform neben dem Sagen annimmt, die Inhalte kommuniziert, die nicht gesagt werden können. Dabei habe ich Wittgenstein so interpretiert, dass er uns durch seine unsinnigen Äußerungen auf diese Inhalte aufmerksam machen will. Eine solche Redeweise würde Conant ablehnen. Die Idee entstammt ihm zufolge der substanziellen Konzeption von Unsinn, welche nur eine dialektische Vorstufe zu Wittgensteins eigentlicher Position ist. Denn ich muss voraussetzen, dass Wittgenstein durch unsinnige Äußerungen wie \enquote{Die Welt zerfällt in Tatsachen.} trotzdem den (logisch verbotenen) Inhalt kommuniziert, dass die Welt in Tatsachen zerfällt, auch, wenn dieser sich eigentlich nur durch Zeigen kommunizieren lässt. Stattdessen will Conant überhaupt keine logisch verbotenen Inhalte annehmen. Unsinn ist einfach-nur-Unsinn. Damit muss weder erklärt werden, wie sich eine besondere Art von Inhalt zeigt und dadurch die Grenzen des Sagbaren überschreitet, noch wie Wittgenstein durch einleuchtenden Unsinn einen solchen Inhalt kommuniziert. Wittgensteins Unsinn kommuniziert nicht, indem er einen unaussprechlichen Inhalt übermittelt, sondern indem er ein Beispiel ist für eine bestimmte Art von Sprachgebrauch, die eben nicht sinnvoll verständlich ist.

Auf das \textbf{Problem der Selbstwiderlegung des Philosophieskeptizismus} kann Conant eingehen, indem er einfach zugesteht, dass der TLP selbstwiderlegend ist. Wenn wir ihn bis zum Ende lesen und die letzte Stufe der dialektischen Leiter erklimmen, sollen wir ihn schließlich ganz von uns stoßen. Er ist Unsinn und gemeinsam mit der restlichen Philosophie zurückzuweisen. Das ist aber kein Problem, weil der Wert des TLP nicht darin besteht, dass er eine systematische und konsistente Argumentation formuliert. Er soll uns eine bestimmte Art von Erfahrung mitgeben, bei der wir dadurch lernen, dass wir erst getäuscht und dann von dieser Täuschung befreit werden. So ist auch das \textbf{Problem der argumentativen Struktur des TLP} gelöst. Der TLP erweckt nur den Anschein einer argumentativen Struktur. Letztendlich funktioniert er aber nicht durch Argumente, sondern durch die eben geschilderte spezielle Form von Täuschungs-Dialektik.

An dieser Interpretation gibt es einiges auszusetzen.\footnote{Hacker behauptet, dass Conant seine Position als Strohmann darstellt. Ich gehe auf die Diskussion zwischen den beiden hier nicht tiefer ein, weil wir schon im Teil zu Hacker sehen konnten, dass dessen Interpretation von Wittgenstein die Leiter-Probleme nicht befriedigend lösen kann. Ich brauche Conants Kritik an Hacker also nicht, um festzustellen, dass für meine systematische Frage danach, ob Wittgenstein uns erklären kann, wo die Grenzen unserer Sprache liegen, Hackers Interpretation ein Indiz ist, dass Wittgenstein das nicht kann. Sie dient mir lediglich dazu, genauer herauszuarbeiten, was Conants Interpretation von Wittgenstein ist, damit ich prüfen kann, ob diese die Leiter-Probleme löst. Vgl. für Hackers Strohmann-Vorwurf \cite{hacker2003}.} Ein Problem ist, dass Conant Wittgenstein, einem der einflussreichsten analytischen Philosophen des 20. Jahrhunderts unterstellen muss, dass große und wesentliche Teile eines seines Hauptwerks absichtlich irreführend und täuschend sind.\autocite[Vgl. für eine ähnliche Kritik an einer Täuschungs-Lesart][151]{moyal-sharrock2007} Damit ist der Wert der geschilderten Tätigkeit zweifelhaft. Denn die Art von Täuschungs-Dialektik, die Conant beschreibt, befreit vielleicht die ein oder andere von der Versuchung, theoretische Philosophie zu betreiben. Aber die meisten Menschen neigen ohnehin nicht dazu, Thesen, wie die, dass die Sprache die Welt abbildet, zu formulieren! Für die meisten Menschen wäre der TLP nach dieser Lesart also nicht nur uninteressant, er wäre sogar schädlich, weil die Gefahr bestünde, dass sie es nicht bis zum Schluss schaffen und bei den absichtlichen Täuschungen hängen bleiben. Die rationale Konsequenz von Conants Interpretation wäre also, dass wir die meisten Menschen aktiv davon abhalten sollten, den TLP zu lesen, anstatt ihn zu empfehlen. Der TLP sollte nur von Menschen gelesen werden, die ohnehin dazu neigen, Wittgenstein-artige Theorien aufzustellen. Und sein Nutzen für diese Menschen wäre nicht, dass er ihnen etwas Neues über philosophische Themen vermittelt, sondern nur, dass er sie zu der Erkenntnis bringt, dass sie endlich aufhören sollten zu philosophieren. Das Werk, das bei Hacker noch ambitioniert, großartig aber letztendlich gescheitert war, wird bei Conant fade, trist und klein. Das macht Conants Interpretation unplausibler, weil wir so auch Wittgensteins wichtige Rolle in der philosophischen Geistesgeschichte schwerer erklären können.

Vor allem kämpft Conants Interpretation mit dem \textbf{Problem der argumentativen Struktur des TLP}. Denn Wittgenstein mag glauben, dass er uns zuerst zu einer substanziellen Auffassung von Unsinn täuscht und uns dann dazu bringt, die Welt richtig zu sehen. All dieses Gerede über Täuschungsversuche und dialektische Vorstufen verrät uns aber allerhöchstens, was Wittgenstein über sein Werk denkt. Es sagt uns nichts darüber, ob wir gute Gründe haben, diese Gedanken zu akzeptieren. In Sekten werden neuen Mitgliedern die eigenen Glaubenssätze häufig nur häppchenweise verkauft. Bei einem frühen Scientology-Treffen erfährt man vielleicht schon, dass die eigenen praktischen Lebensprobleme durch kleine böse Geister verursacht werden. Man lernt aber noch nicht, dass der Sektenanführer persönlich gegen außerirdische Aliens gekämpft hat. Ältere Sektenmitglieder täuschen neuere Sektenmitglieder also bewusst (Es geht nur um Geister), um diese Täuschung später aufzulösen und ihnen zu helfen, die Welt richtig zu sehen (Eigentlich geht es um Aliens). An diesem Beispiel sieht man aber, dass so ein Theater nicht ohne eine argumentative Stützung der Art und Weise, die Welt richtig zu sehen, auskommt.\autocite[Für eine Diskussion verschiedener Sekten und ihrer Struktur verschiedener \enquote{Ebenen der Wahrheit} vgl.][]{hassan2015} Wenn das bei Sekten so ist, gilt es aber auch für Wittgenstein. Damit wir akzeptieren können, dass die Art und Weise, auf die wir nach Wittgenstein die Welt nach der Auflösung seiner Täuschung sehen sollen, auch die richtige Art ist, die Welt zu sehen, muss er Argumente vorbringen. Ich will hier nicht bestreiten, dass es solche Argumente gibt. Mithilfe von Conants Interpretation des Kontextprinzips lassen sich sicherlich manche konstruieren. Das Problem ist aber, dass solche Argumente, damit wir sie als \emph{Argumente} akzeptieren können, aus Prämissen und Konklusionen bestehen müssen, die in einem logisch gültigen Schluss (oder in einem formal korrekten induktiven Argument) zusammenhängen. So etwas darf Wittgenstein aber nach der Interpretation, die uns Conant schildert, höchstens verwenden, um uns zu täuschen, denn die Welt richtig zu sehen, lernen wir nach dieser Interpretation ja nur durch \enquote{einfach-nur-Unsinn}. Und in \enquote{einfach-nur-Unsinn} ist eine logische Struktur, wie sie für ein Argument nötig wäre, nicht erkennbar.

Obwohl Conant es schafft, mit den Problemen der \textbf{Selbstwiderlegung des Philosohieskeptizismus}, der \textbf{Grenzen des Zeigbaren} und der \textbf{Kommunikation des TLP} umzugehen, gelingt es ihm also nicht, auf Wittgensteins Leiter zu klettern, weil der TLP bei ihm auf unplausible Weise täuscht. Vor allem und schwerwiegernder scheitert sein Ansatz aber an dem \textbf{Problem der argumentativen Struktur des TLP}.


% Vertritt Hacker das: Auch diese Rekonstruktion ist aber nicht außerordentlich präzise. Hacker schreibt zwar, dass mancher Unsinn in Verletzungen der Regeln logischer Syntax besteht, ansonsten würde er die Positionen, die ihm hier unterstellt werden, aber vermutlich ablehnen. Bei Conants \emph{Merkmalen der Wirklichkeit} müsste man sich zuerst fragen, ob \enquote{Wirklichkeit} im wittgensteinschen oder in einem anderen Sinn verwendet wird... % TODO: HIER WEITER!!!

% \begin{quote}
%   Illuminating nonsense will guide the attentive hearer or reader to apprehend what is shown by propositions that do not purport to be philosphical; moreover it will intimate, to those who grasp what is meant, its own illegitimacy.\autocite[][18]{hacker1972}
% \end{quote}

% Hacker grenzt im darauf folgenden Abschnitt einleuchtenden Unsinn dann nicht dadurch von irreführendem Unsinn ab, dass dieser zeigt, was nicht gesagt werden kann, sondern dadurch, dass dieser aufmerksame LeserInnen dazu leiten wird, zu erkennen, was sinnvolle Sätze zeigen.\autocite[Vgl.][18]{hacker1972}



% Später: schwächere Version: \claim{Kontextprinzip, schwach} Eine Satzkomponente hat nur einen Sinn, wenn sie in einem sinnvollen Satz vorkommen kann.
% \claim{Kontextprinzip, stark} Eine Satzkomponente hat nur einen Sinn, wenn sie in einem sinnvollen Satz vorkommt.


\subsection{Moyal-Sharrocks Zwischenweg}

Danièle Moyal-Sharrock versucht, aus den Lagern der substanziellen und resoluten Lesarten auszubrechen und eine alternative, dritte Lesart zu begründen. Sie sieht auf beiden Seiten Stärken und Schwächen. Ich bin schon im Conant-Teil auf die Probleme eingegangen, die sie bei der resoluten Lesart entdeckt. Durch die spezielle Art von Täuschungs-Dialektik, die Conant Wittgenstein unterstellt, muss er große Teile dessen Werks als irreführende Täuschung ablehnen. Insofern ist der Inhalt des TLP selbstablehnend (self-repudiating). Moyal-Sharrock argumentiert zusätzlich dafür, dass die resolute Lesart nicht erklären kann, wie es einfach-nur-Unsinn gelingen soll, Erkenntnisse zu vermitteln. Sie unterstellt Conant also, dass dieser das \textbf{Problem der Kommunikation} des TLP nicht lösen kann. Hier haben wir aber schon gesehen, dass es Conant gelingt, auf dieses Problem einzugehen. Genau so wenig, wie eine Karikatur explizit sprachlich auf das Bezug nehmen muss, was sie durch überspitzte Darstellung kritisiert, muss das Wittgensteins TLP.\autocite[Vgl.][Abschnitt 1.1: The Therapeutic Circle and Therapeutic Nonsense]{moyal-sharrock2007}

An der substanziellen Lesart kritisiert Moyal-Sharrock im Gegenzug, dass sie zwar dem Inhalt aber nicht der Form von Wittgensteins Werk gerecht werden kann. Denn Hacker zufolge versucht Wittgenstein mit dem TLP, unaussprechliche metaphysische Wahrheiten auszudrücken, obwohl das nach seiner eigenen philosophischen Methode streng genommen nicht geht. Damit würde er aber seinem Anspruch an Klarheit widersprechen. Denn Wittgenstein verrät uns schließlich Vorwort, dass der Wert seiner Arbeit darin besteht, \enquote{daß in ihr Gedanken ausgedrückt sind, und dieser Wert wird umso größer sein, je besser die Gedanken ausgedrückt sind.}\footnote{\cite{wittgenstein1922}, Vorwort.} Außerdem schreibt er, dass sein Projekt gelingt und zeigt diese Auffassung auch in seinen Handlungen, indem er die Philosophie für ein Jahrzehnt ruhen lässt. Moyal-Sharrock nimmt auch an, dass Wittgenstein, anders als Hacker ihn versteht, keine metaphysischen Thesen vertritt. Diese Auffassung begründet sie aber nicht ausführlich, sondern beruft sich nur auf Norman Malcolm.\footnote{Vgl. \cite{malcolm1986}, viii. Zitiert nach \cite{moyal-sharrock2007}, S. 154.} Vor allem kritisiert Moyal-Sharrock an der substanziellen Lesart aber, dass ihrzufolge Wittgenstein mit seinem Versuch, die Grenzen von Sprache zu ziehen, diese überschreitet und dass er nicht seiner eigenen streng genommenen philosophischen Methode folgt. Insofern ist die Form des TLP selbstablehnend.\autocite[Vgl.][Abschnitt 1.2: Peter Hacker and illuminating nonsense]{moyal-sharrock2007}

Entgegen dieser Lesarten will Moyal-Sharrock sowohl der Form als auch dem Inhalt von Wittgensteins Philosophie gerecht werden. Ihr zentraler Argumantationsschritt ist hier, dass sie \enquote{Unsinn} auf unterscheidende (discriminatory) und nicht auf abwertende (pejorative) Weise versteht. Durch den Begriff weist Wittgenstein eine Position nicht immer als illegitim auf. Es ist einfach nur ein technischer Begriff, den er verwendet, um zwischen verschiedenen Arten von Sätzen zu unterscheiden. Dabei stellt sie vier Arten von Unsinn fest:

\begin{enumerate}
  \item \textbf{Taugologie und Kontradiktion:} \enquote{Wenn es regnet und wenn die Straße immer nass ist, wenn es regnet, dann ist die Straße nass.}
  \item \textbf{Gibberish:} Sätze denen nie ein Sinn verliehen wurde: \enquote{Er ist ein Zalunkel!}
  \item \textbf{Kategorienfehler:} Sätze die durch inkompatible logische Kategorien gegen die Regeln der logischen Syntax verstoßen: \enquote{Sokrates ist identischer als das Schöne.}\footnote{Angelehnt an \cite{wittgenstein1922}, 5.473 und 5.4733.}
  \item \textbf{Grammatische Regel:} Sätze, die die Grenzen des Sinns demarkieren: \enquote{Die im Satz angewandten einfachen Zeichen heißen Namen.}\footnote{\cite{wittgenstein1922}, 3.202.}
\end{enumerate}

Denn dass ein Satz \enquote{unsinnig} ist, heißt nur, dass ihm ein Sinn fehlt. Und das kann verschiedene Gründe haben. Demnach nimmt Moyal-Sharrock auch den Unterschied zwischen sinnlosen und unsinnigen Sätzen nicht so genau. \textbf{Tautologien und Kontradiktionen} sind bei ihr auch Unsinn, da sie eben Sätze sind, denen ein Sinn fehlt. Und Unsinn ist verschieden schlimm. Wenn man in dem Versuch, tiefgründig zu philosophieren, \textbf{Gibberish} oder \textbf{Kategorienfehler} produziert, ist das ein Problem für die eigene Position. \textbf{Grammatische Regeln} haben dagegen keinen Sinn, sondern umgrenzen nur die Möglichkeiten von Sinn und das ist kein Problem.\autocite[Vgl.][Abschnitt 2.1: The different ways of lackings sense in the TLP]{moyal-sharrock2007}\footnote{Hier ist außerdem zu bemerken, dass Moyal-Sharrock mit \textbf{Gibberish} und \textbf{Kategorienfehlern} resolute und substanzielle Verständnisse von Unsinn in ihre Konzeption aufnimmt. Dabei werden beide zu Beispielen von Unsinn, der viele verschiedene Formen annehmen kann.}

Wittgensteins TLP besteht in \textbf{grammatischen Regeln}. Als Beleg für diese Interpretation führt Moyal-Sharrock an, dass Wittgenstein in \emph{Philosophische Grammatik} sowohl grammatische Regeln als auch deren Negation als \enquote{Unsinn} bezeichnet. Er wiederholt diesen Gedanken auch in einem Brief. Und Moore kann sich erinnern, dass Wittgenstein sich in Gesprächen mit \enquote{grammatische Regel}  und \enquote{Unsinn} gleichermaßen auf die Sätze bezieht, die gewöhnlich als \enquote{notwendige Proposition} verstanden werden.\autocite[Vgl.][Abschnitt 2.2: The good sense of nonsense]{moyal-sharrock2007} Moyal-Sharrock zufolge besteht der TLP also weder in Behauptungen noch in Vorschriften. Wittgenstein teilt uns Erläuterungen durch \textbf{grammatische Regeln} mit, die helfen, die Bedingungen für sinnvollen Diskurs übersichtlich zu machen. Er schlägt zwar vor, dass das in Zukunft am Besten an konkreten Beispielen von einzelnen Sätzen geschehen soll, doch damit widerspricht er nicht seiner eigenen Methode. Denn dieser Vorschlag würde sich für ein gesamtes Buch einfach nicht eignen.\autocite[Vgl.][Abschnitt 3: Tractarian \enquote{propositions} as expressions of the bounds of sense]{moyal-sharrock2007}

Bei all dem stellt sich immer noch die Frage, wie der Teil des \textbf{Leiter-Satzes}, in dem Wittgenstein empfiehlt, die Leiter umzustoßen und seine Forderung, über Unaussprechliches zu schweigen, zu verstehen sind. Schließlich scheint er, indem er Unsinn äußert, trotzdem von etwas zu sprechen, von dem man nicht sprechen kann. Hier schlüssellt Moyal-Sharrock Wittgensteins Aussagen auf, indem sie zwischen \enquote{sagen} und \enquote{sprechen} unterscheidet. \enquote{Sagen} ist ein technischer Ausdruck, unter den nur sinnvolle Äußerungen fallen. Wir können aber durchaus Unsinn \enquote{sprechen}, der die Grenzen des Sagbaren demarkiert:

\begin{quote}
  I suggest that we need not resort to irony to avoid saddling Wittgenstein with inconsistency or blunder---or worse: paradox. Rather, we must acknowledge that Wittgenstein uses \emph{saying} in a specialized, technical sense. If we make a distinction between Wittgenstein's technical use of \emph{saying}, and his nontechnical use---for which we might substitute the verbs: \emph{speaking, articulating, voicing or uttering}---the incinsistency vanishes. So that allthough Tractarian sentencecs cannot (technically) be \emph{said}, this does not mean they cannot be \emph{voiced} or \emph{spoken} or \emph{uttered}.\footnote{\textbf{Deutsch:} I schlage vor, dass wir nicht auf Ironie zurückgreifen müssen, um zu vermeiden Wittgenstein mit Inkonsistenz oder Fehlern---oder schlimmer: Paradoxien---zu belasten. Wenn wir eine Unterscheidung zwischen Wittgensteins technischem Gebrauch von \emph{sagen} und seinem nicht-technischen Gebrauch---den wir durch die Verben \emph{sprechen, artikulieren oder äußern} ersetzen könnten---einführen, verschwindet die Inkonsistenz. Obwohl die Sätze des TLP also (technisch gesehen) nicht \emph{gesagt} werden können, heißt das nicht, dass sie nicht \emph{artikuliert} oder \emph{gesprochen} oder \emph{geäußert} werden können. (Für \enquote{voicing} gibt es in diesem Kontext leider keine passende deutsche Übersetzung.) \cite{moyal-sharrock2007}, S. 171-172.}
\end{quote}

Wittgensteins TLP soll also nicht mehr die Grenzen aller Sprache ziehen, sondern nur die einer speziellen Verwendung von Sprache, die sie als \enquote{sagen} bezeichnet. Andere Arten Sprache zu verwenden bleiben unbegrenzt.\autocite[Vgl.][Abschnitt 4: Throwing away the ladder; insbesondere 4.1: Saying is use]{moyal-sharrock2007} Wittgenstein schreibt zwar: \enquote{Wovon man nicht \emph{sprechen} kann, darüber muß man schweigen.}\footnote{\cite{wittgenstein1922}, 7.} Er beendet also sein Werk mit einem Satz, der explizit der Art und Weise \enquote{sagen} und \enquote{sprechen} zu verstehen widerspricht, die Moyal-Sharrock ihm unterstellt. Doch davon lässt sich sie sich nicht beirren. Er folge seiner präzisen technischen Verwendung eben nur manchmal und wenn er es nicht tue, müssten wir nachhelfen.\autocite[Vgl.][Fußnote auf S. 172]{moyal-sharrock2007}

Damit wird das \textbf{Problem der Kommunikation} des TLP gelöst. Es bleibt zwar rätselhaft, wie es dem TLP gelingt, durch Unsinn zu kommunizieren. Aber jetzt vertritt Wittgenstein auch nicht mehr den Anspruch, alle sprachliche Kommunikation zu erklären, denn es geht ihm ja nur noch ums Sagen. Das erklärt auch das \textbf{Problem der Grenzen des Zeigbaren}. Wittgenstein untersucht diese nicht einmal, da es ihm nur um die Grenzen des Sagens geht. Auch Wittgensteins \textbf{Philosophieskeptizismus} ist nicht mehr selbstwiderlegend. Denn \enquote{Unsinn} ist jetzt ein technischer Begriff, der eine Position nicht notwendig abwertet. Und das gelingt sogar, ohne dass der Schlange, die sich selbst in den Schwanz biss, dabei ihre Zähne gezogen werden. Denn da Moyal-Sharrock neben \textbf{grammatischen Regeln} noch andere Arten von Unsinn zulässt, von denen \textbf{Gibberish} und \textbf{Kategorienfehler} trotzdem ein Grund sind, Positionen abzulehnen, kann Wittgenstein selbst Unsinn reden und immer noch den Unsinn anderer kritisieren.

Das \textbf{Problem der argumentativen Struktur des TLP} kann Moyal-Sharrock nicht umgehen. Denn selbst wenn der TLP in \textbf{grammatischen Regeln} besteht, brauchen wir Argumente, um zu zeigen, dass diese Regeln plausibel sind. Schließlich haben wir es hier nicht mit einem Deutschlehrer zu tun, der seinen SchülerInnen die deutsche Sprache mit autoritärer Gewissheit lehrt. Wittgenstein ist ein Philosoph neben vielen anderen und es steht ja gerade zur Debatte, ob die Regeln, die er aufstellt, gelten, oder nicht. Hier kann Moyal-Sharrock Wittgensteins Position nur retten, wenn sie annimmt, dass Wittgenstein sich, indem er seine eigenen Worte als Unsinn bezeichnet, nur auf wenige \textbf{grammatische Regeln} bezieht, die von vielen sinnvollen Argumenten umrahmt werden.

Insgesamt klingt das beinahe nach einer guten Bilanz. Doch leider ist Moyal-Sharrocks Ansatz zirkulär. Denn sie kann die Leiter-Probleme nur lösen, indem sie annimmt, dass wir Unsinn in \textbf{Gibberish}, \textbf{Kategorienfehler} und \textbf{grammatische Regeln} aufteilen können. Doch die Leiter-Probleme entstehen erst, weil aus Wittgensteins Ansatz folgt, dass jeder Versuch, eine \textbf{grammatische Regel} zu artikulieren, \textbf{Gibberish} bzw. ein \textbf{Kategorienfehler} ist! Ein Satz wie \enquote{Die Welt zerfällt in Tatsachen.}\footnote{\cite{wittgenstein1922}, 1.2.} ist für Conant \textbf{Gibberish}, weil es keine Interpretation gibt, bei der wir allen Komponenten des Satzes einen Sinn verleihen können. Für Hacker ist es ein \textbf{Kategorienfehler}, weil die Kategorien \enquote{Welt} und \enquote{Tatsache} es nicht erlauben, sie in dieser Weise zu kombinieren. Indem Moyal-Sharrock den Satz aber als \textbf{grammatische Regel}, die die Grenzen des Sinns aufweist, liest, setzt sie schon voraus, was zur Debatte steht: Dass das möglich ist, ohne dass die \textbf{grammatische Regel} dabei auf eine problematische Art unsinnig ist.

Dazu kommt, dass Wittgenstein nach Moyal-Sharrocks Interpretation nicht mehr seine eigentliche Fragestellung nach den Grenzen unserer Sprache beantwortet. Nach ihr geht es ihm nur um Grenzen des Sagens, diese können aber durch andere Arten des Äußerns, Artikulierens und Sprechens überschritten werden. Hier stellt sich nicht nur die Frage, was \enquote{sagen} dann genau ist und warum seine Grenzen interessant sind. Vor allem ist dabei die Frage, um die es hier geht, nicht einmal berührt. Damit ist Moyal-Sharrocks Ansatz, selbst wenn er eine plausible Interpretation wäre, für das Projekt, das ich hier verfolge, nicht brauchbar.

\section{Fazit}

Ich fasse also die Ergebnisse meiner Arbeit zusammen. Meine Kernfrage war, wo die Grenzen unserer Sprache liegen. Genauer habe ich diese Frage ergründet, indem ich rekonstruiert habe, wie Wittgenstein sie im TLP beantwortet und indem ich geprüft habe, ob diese Antwort plausibel ist.

Dazu war es zuerst nötig, Wittgenstein in seinen historischen Kontext einzuordnen. Ich habe mit Frege und Russell zwei Vordenker Wittgensteins geschildert, die auch mit den Grenzen von Sprachen zu kämpfen haben. Doch anders als Wittgenstein treten Frege und Russell nicht mit dem Anspruch auf, die Grenzen von Sprache schlechthin zu erkunden. Sie formulieren Idealsprachen, die wie Mikroskope gedacht sind. Sie sind für viele Zwecke ungeeignet, aber sie erlauben es uns, manches klarer zu sehen, das wir sonst nie entdeckt hätten. Beide Autoren können so dem Problem, dass ihre Idealsprachen sich nicht selbst erläutern können, aus dem Weg gehen. Dazu waren sie ohnehin nie gedacht.

Doch Wittgenstein hat größere Ambitionen. Um \enquote{die philosophischen Probleme} \enquote{im Wesentlichen und endgültig} zu lösen, will er zeigen, dass ihre Fragestellung \enquote{auf dem Mißverständnis der Logik unserer Sprache} beruht.\footnote{Vgl. \cite{wittgenstein1922}, Vorwort.} Philosophische Fragen sind demnach Unsinn und als solcher zurückzuweisen.

Ich habe also die elaborierte Struktur unserer Sprache geschildert, die Wittgenstein beschreibt, um die philosophischen Probleme zu lösen. Jeder Satz besteht aus Elementarsätzen, die durch Wahrheitsfunktionen verknüpft werden. Diese Elementarsätze setzen Namen zueinander in ein Verhältnis. Sie beziehen sich auf die Welt, da diese aus Tatsachen besteht und jeder Elementarsatz mit einem Sachverhalt übereinstimmt, der, wenn er besteht, eine Tatsache ist. Hier bin ich wieder auf das Problem gestoßen, das wir schon bei Frege und Russell kennen gelernt hatten: Die Idealsprache lässt sich nicht in der Idealsprache erläutern.

Bei Wittgenstein stellte sich dieses Problem als schwerwiegender heraus. Mit seiner Unterscheidung zwischen sagen und zeigen und dem \textbf{Leiter-Satz} geht er auf das Problem ein. Doch trotz dieser Ausführungen konnte ich vier \textbf{Leiter-Probleme} feststellen, aufgrund derer sein Ansatz als Versuch, die Grenzen unserer Sprache zu ergründen, unplausibel erschien. Wittgensteins Ansatz ist erstens selbstwiderlegend, weil er unsinnig ist, die restliche Philosophie aber aufgrund ihrer Unsinnigkeit kritisiert. Zweitens kann er die Grenzen der Sprache nur dadurch ziehen, dass er die Grenzen einer weiteren Ausdrucksform, der des Zeigens offenlässt. Drittens bleibt rätselhaft, wie Wittgenstein durch Unsinn, der weder etwas sagt noch etwas zeigt, kommuniziert. Und viertens kann Wittgenstein, indem er Unsinn schreibt, keine Argumente formulieren, er kann also keine Gründe dafür geben, dass sein Ansatz plausibel ist.

Ich habe dann verschiedene Interpretationen des TLP daraufhin geprüft, ob sie es schaffen, alle \textbf{Leiter-Probleme} zu lösen. Hackers substanzielle Lesart geht auf die Probleme der \textbf{Selbstwiderlegung des Philosophie-Skeptizismus} und der \textbf{Grenzen des Zeigbaren} ein. Doch sein Ansatz vermeidet Wittgensteins Selbstwiderlegung nur um den Preis, dass dieser auch die restliche Philosophie nicht mehr widerlegt. Das Problem der \textbf{Grenzen des Zeigbaren} ist für Hacker selbst Grund, Wittgensteins TLP zurückzuweisen. Dazu kommen die Probleme der \textbf{Kommunikation des TLP} und der \textbf{Argumentativen Struktur des TLP}, auf die er nicht eingeht.

Conant gelingt es, die Probleme der \textbf{Argumentativen Struktur}, der \textbf{Grenzen des Zeigbaren} und der \textbf{Kommunikation des TLP} zu beantworten. Leider muss er dafür Wittgensteins TLP in wesentlichen Teilen als absichtliche Täuschung verstehen, was unplausibel ist. Außerdem kann er das \textbf{Problem der argumentativen Struktur des TLP} nicht lösen.

Auch Moyal-Sharrock schafft es nicht, das \textbf{Problem der argumentativen Struktur des TLP} zu lösen. Von allen Leiter-Problemen stellte sich dieses also als das hartnäckigste heraus. Es schien erst, als könnte sie die anderen Probleme lösen. Doch letztendlich scheitert ihr Ansatz an einer zirkulären Argumentation. Selbst wenn man diese Probleme außen vor lässt, musste ich ihn in diesem Rahmen zurückweisen, weil sie Wittgenstein eine andere Fragestellung unterstellt als die, die ich hier untersucht habe.

Wenn alle betrachteten Ansätze scheitern, könnte das ein Zeichen dafür sein, dass etwas mit den angelegten Maßstäben nicht stimmt. Vielleicht ist all das ein Indiz, dass meine Frage schon falsch gestellt war. Vielleicht gibt es keine Grenzen unserer Sprache. Vielleicht gibt es Grenzen unserer Sprache, doch sie lassen sich nicht sinnvoll erläutern. Vielleicht lassen sie sich nur \emph{in unserer Sprache} nicht sinnvoll erläutern und auf andere Weise schon. Das sind Fragen und Vermutungen, die in zukünftigen Arbeiten ergründet werden könnten. Für diese Arbeit kann ich nur festhalten, dass Wittgensteins TLP nach allen hier geprüften Interpretationen keinen plausiblen Ansatz bietet, die Grenzen unserer Sprache zu verstehen.


% Literaturverzeichnis

% \nocite{hab_euch_benutzt_aber_nicht_zitiert}

% Sängerin: Bild von <a href="https://pixabay.com/de/users/MirceaIancu-11873433/?utm_source=link-attribution&amp;utm_medium=referral&amp;utm_campaign=image&amp;utm_content=4324839">Mircea Iancu</a> auf <a href="https://pixabay.com/de/?utm_source=link-attribution&amp;utm_medium=referral&amp;utm_campaign=image&amp;utm_content=4324839">Pixabay</a>

% Pegasus: Bild von <a href="https://pixabay.com/de/users/KELLEPICS-4893063/?utm_source=link-attribution&amp;utm_medium=referral&amp;utm_campaign=image&amp;utm_content=3395135">Stefan Keller</a> auf <a href="https://pixabay.com/de/?utm_source=link-attribution&amp;utm_medium=referral&amp;utm_campaign=image&amp;utm_content=3395135">Pixabay</a>

\nocite{mcginn2006}
\nocite{gellman2019}
\nocite{kienzler2008}
\nocite{glock2015}
\nocite{ricketts2017}
\nocite{carnap1931}
\nocite{irvine2020a}



\clearpage
\printbibliography

% Eigenständigkeitserklärung
\clearpage

\section*{Eigenständigkeitserklärung}

Hiermit versichere ich, dass ich diese Arbeit mit Ausnahme von dieser Erklärung ausschließlich mit den angegebenen Hilfsmitteln verfasst habe. Technische Hilfsmittel wie Texteditoren, Zitationsverwaltungssoftware, etc. sind hiervon ausgenommen. Alle sonstigen Passagen, die ich wörtlich oder sinngemäß aus Büchern, Artikeln oder anderen Quellen übernommen habe, habe ich als Zitate kenntlich gemacht. Darüber hinaus versichere ich, dass diese Arbeit nicht schon Gegenstand eines anderen Prüfungsverfahrens gewesen ist.

\end{document}



% Erste Beispiele:

% Mystizismus: Die Mystische Erfahrung oder ihre Gegenstände

% - Fallbeispiel: Daoismus? Zen?
%     - Zen und PCEs?

% Individuelle phänomenale Erfahrung? Qualia?

% Unwahrnehmbares:

% - Materie abgesehen von unseren Wahrnehmungen (Berkeley)
% - Der Geist in meiner Uhr (Rosenberg)

% Prinzipiell unverifizierbares:

% - Welt erst vor 5min entstanden
% - Was vor dem Urknall passiert ist
% - Paralleluniversen



% Spätere Beispiele:

% Frege: Logik erklären

% Heidegger: Logik missachten

% Ethik, Ästhetik, Metaphysik
