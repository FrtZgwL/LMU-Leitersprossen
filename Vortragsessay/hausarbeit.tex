% Lügen und und durch Fiktion - Cedric Tonye-Djon

% Grundeinstellungen, Formatierung, Ränder, Schrift, Zitierstil, usw.
\documentclass[fontsize=12pt, paper=a4, headings=standardclasses, parskip=half, bibliography=numbered]{scrartcl}
\usepackage[left=3cm,right=3cm,top=3cm,bottom=3cm]{geometry}
\usepackage[rm]{libertine}
\usepackage[english,german,main=ngerman]{babel}
\usepackage[autostyle=true,german=quotes,autopunct=true]{csquotes}
\usepackage[backend=biber,style=authoryear,autocite=footnote]{biblatex}
\addbibresource{E:\\OneDrive - bwedu\\Zotero\\Meine Bibliothek.bib}
\deffootnote{1.5em}{1em}{%
  \makebox[1.5em][l]{\thefootnotemark}%
}
\usepackage{fnpct}
\AtBeginEnvironment{quote}{\small}

\makeatletter
\newcommand{\claim}{%
  \@startsection{paragraph}{4}%
  {\z@}{0em \@plus 1ex \@minus .2ex}{-1em}%
  {\normalfont\normalsize\bfseries}%
}
\makeatother

  % Gerstorfers Zeug, von dem ich nicht weiß, wozu es gut ist
\usepackage[unicode=true,pdfborder={0 0 0},hyperfootnotes=false]{hyperref}
\usepackage{bookmark}
\usepackage[onehalfspacing]{setspace}
\usepackage{hyperref}


\begin{document}

% Deckblatt
\begin{titlepage}
Universität Stuttgart\\
Institut für Philosophie\\
Seminar\\
DozentIn\\\\
Modul\\
Seiten (1 Normseite = 1800 Zeichen)\\
Gefordert waren Geforderte Seiten Seiten\\
\vfill
\begin{center}
  \Huge 4 Lose Leitersprossen\\
  \large Gelingt Wittgensteins \emph{Tractatus logico-philosophicus} als Kriterium von Unsinn?
\end{center}
\vfill
% Cedric Tonye-Djon\\
Matrikelnummer: 3224391\\
cedric.tonyedjon@gmail.com\\
1-Fach-Bachelor Philosophie\\
Semester. Fachsemester
\end{titlepage}

%Inhaltsverzeichnis
\tableofcontents
\clearpage


\section{Offene Fragen}

\begin{enumerate}
  \item Ist "Kriterium" das richtige Wort?
\end{enumerate}

\section{Einleitung}

Besonders in der Philosophie drehen sich Debatten häufig darum, ob gewisse Formulierungen überhaupt sinnvolle Sprache oder ob sie nichts weiter als Geräusche, Tintenkleckse, leuchtende Pixel, Unsinn sind. \textbf{Das ist relevant, weil der Nachweis von Unsinnigkeit eine philosophische Position als solche widerlegt.} Ludwig Wittgenstein versucht mit dem \emph{Tractatus logico-philosophicus} (im Folgenden TLP) unter anderem ein allgemeines Kriterium für Unsinn zu entwickeln. In diesem Essay werde ich diesen Versuch diskutieren und dabei feststellen, dass er als Unsinns-Kriterium scheitert. Dabei betrachte ich nur die Interpretation James Conants.

\section{Rekonstruktion des TLP}

Beginnen wir mit einer groben Rekonstruktion des TLP. Conant zufolge sind ein Großteil der Überlegungen des TLP nur eine \enquote{dialektische Vorstufe} zu dessen eigentlicher Position. Ich werde also zuerst die Überlegungen naiv schildern und dann darauf eingehen, inwiefern sie Conant zufolge später aufgehoben werden.

Wittgenstein interessiert sich für ein Kriterium, das zwischen sinnvoller und unsinniger Sprache unterscheidet, weil die Unsinnigkeit einer philosophischen Position für ihn Grund genug ist, sie in folgender Hinsicht zurückzuweisen: Sie soll nicht mehr artikuliert werden und Argumente für oder gegen sie sind hinfällig. Denn so lässt sich am besten erklären, wieso Wittgenstein der Philosophie den Rücken kehrte, nachdem er \enquote{die philosophischen Probleme} \enquote{im Wesentlichen endgültig [löste]}.\autocite[Vgl.][Vorwort]{wittgenstein2016}\textbf{Lebenslauf-Zitat} Zudem schreibt er zum Schluss des TLP explizit: \enquote{Wovon man nicht sprechen kann, darüber muß man schweigen.}\autocite[][7]{wittgenstein2016}

Um herauszufinden, welche Sätze unsinnig sind, untersucht Wittgenstein die Bedingungen der Möglichkeit von Sinn. Dabei vertritt er anfangs, dass sinnvolle Sprache nur dadurch möglich ist, dass Sprache bestimmten Strukturen unterliegt, die in in einer formallogischen Idealsprache besonders offensichtlich erkennbar sind. Sinnvolle Sprache besteht aus Namen, die jeweils mit Gegenständen übereinstimmen. Diese Namen müssen in Elementarsätzen miteinander verbunden werden und so stimmen sie mit Sachverhalten überein, in denen Gegenstände miteinander verbunden sind. Diese Struktur zeigt sich beispielsweise in Sätzen wie $aRb$, in denen $a$ und $b$ als Namen funktionieren, die im Satz auf eine bestimmte Art verbunden sind. Sachverhalte können bestehen oder nicht und wenn ein Satzverhalt besteht, ist der Satz, der mit ihm übereinstimmt, wahr. Elementarsätze und komplexe Sätze lassen sich durch Wahrheitsfunktionen wie \enquote{und} und \enquote{oder} zu komplexen Sätzen kombinieren. Dabei legt die Wahrheitsfunktion fest, mit welcher Kombination des Bestehens und nicht-Bestehens von Sachverhalten der resultierende Satz übereinstimmt. Wenn $p$ und $q$ beispielsweise Elementarsätze sind, stimmt $p \wedge q$ nur damit überein, dass der Sachverhalt, der mit $p$ übereinstimmt und der, der mit $q$ übereinstimmt, besteht. $p \vee q$ stimmt dagegen mit dieser Wahrheitsmöglichkeit und noch zwei weiteren überein. \enquote{Sinn} ist nach Wittgenstein nichts weiter als dieses Übereinstimmen und nicht Übereinstimmen. Sätze wie $a \vee a$ oder $a \wedge \sim a$ sind demnach \emph{sinnlos}, weil sie mit allen oder keinen ihrer Wahrheitsmöglichkeiten übereinstimmen. Sätze wie $a \vee \wedge \neg$, \enquote{Das Nichts selbst nichtet.} oder \enquote{Du sollst nicht töten!} sind \emph{unsinnig}, weil sie sich nicht in Namen, Elementarsätze und Wahrheitsfunktionen zerlegen lassen.\autocite[][]{heidegger-zitiert-nach-carnap} Damit sind Wittgensteins Versuche, die Strukturen seiner Idealsprache aufzuweisen, aber selbst unsinnig. Das merkt man, wenn man versucht sich vorzustellen, wie ein Satz wie: \enquote{Die Welt zerfällt in Tatsachen.} auf Elementarsätze zurückzuführen wäre.\autocite[][1.2]{wittgenstein2016} Der Satz kann selbst kein Elementarsatzsein. Denn weder die Welt noch die Tatsachen sind Gegenstände, die in einem Sachverhalt verbunden werden. Und hier hilft es auch nicht, den Satz als Komplex von Elementarsätzen zu verstehen. Sprache, wie Wittgenstein sie sich vorstellt, ist nicht dazu geeignet, in dieser Weise etwas über Welt und Tatsachen auszusagen. Wittgenstein erkennt das auch an. Er schreibt:

\begin{quote}
  Meine Sätze erläutern dadurch, daß sie der, welcher mich versteht, am Ende als unsinnig erkennt, wenn er durch sie -- auf ihnen -- über sie hinausgestiegen ist. (Er muß sozusagen die Leiter wegwerfen, nachdem er auf ihr hinaufgestiegen ist.)

  Er muß diese Sätze überwinden, dann sieht er die Welt richtig.\autocite[][6.54]{wittgenstein2016}
\end{quote}

Wittgenstein unterscheidet zwischen sagen und zeigen. Sinnvolle Sätze sagen etwas, indem sie mit Sachverhalten übereinstimmen und nicht übereinstimmen. Zusätzlich zeigt sich in sinnvollen und sinnlosen Sätzen die allgemeine Struktur von Sprache und Wirklichkeit, die Wittgenstein mit \enquote{Name}, \enquote{Elementarsatz}, \enquote{Gegenstand} etc. umschreibt, wenn diese Sätze in einer formallogischen Idealsprache übersichtlich dargestellt werden. Dabei ist es wichtig zu bemerken, dass unsinnige Sätze weder etwas sagen noch etwas zeigen. Denn bei unsinnigen Sätzen liegt keinerlei Übereinstimmung und nicht Übereinstimmung mit Sachverhalten vor. Sie bestehen nicht aus Namen, die in Sätzen verbunden und durch Wahrheitsfunktionen verknüpft werden. Daher lassen sie sich nicht in eine formale Idealsprache übersetzen und es lassen sich an ihnen auch nicht allgemeine Strukturen von Sprache und Wirklichkeit erkennen. Wir dürfen den rätselhaften 6.54 also nicht so verstehen, dass Wittgensteins unsinnige Sätze unaussprechliche Einsichten zeigen. Es sollen unsinnige Sätze sein, die uns helfen zu erkennen, was andere sinnvolle und sinnlose Sätze sagen und zeigen.\footnote{Conant und Peter Hacker teilen diese Kritik an der Auffassung, dass Wittgensteins Unsinn etwas zeigt. Vgl. \cite{conant2002}, Fußnote 4 und \cite{hacker1972}, S. 18.}

Doch daraus ergeben sich einige Fragen. Ich werde hier vier Probleme, vier lose Leitersprossen schildern, die dazu führen könnten, dass wir herabstürzen, wenn wir die Leiter unbedacht erklimmen.

\section{Vier lose Leitersprossen}

Das erste Problem ergibt sich daraus, dass der Nachweis der Unsinnigkeit einer Position Grund genug ist, sie zurückzuweisen. Denn daraus folgt, dass Wittgensteins eigene Position zurückzuweisen ist. Auf diese Weise ist sein Projekt selbstwiderlegend.

Zweitens ist ein Unsinns-Kriterium gerade deshalb interessant, weil es eine Grenze zwischen vertretbaren und nicht artikulierbaren philosophischen Positionen zieht. Wenn es aber neben dem Sagen mit Zeigen noch eine weitere Art gibt, auf die Sätze etwas kommunizieren können, dann ist diese Grenze noch nicht gezogen. Heidegger könnte beispielsweise darauf beharren, dass ein Satz wie \enquote{Das Nichts selbst nichtet.} zwar Unsinn sei, dass sich das, auf das uns dieser Satz hinzuweisen versucht, aber in unseren anderen Sätzen über das Nichts zeige.

Drittens stellt sich die Frage, wie Wittgenstein durch seinen Unsinn mit uns kommuniziert. Schließlich unterscheiden sich sprachliche Äußerungen, durch die wir kommunizieren und einander verstehen können, eben dadurch von bloßen Geräuschen, dass sie einen Sinn haben. Wenn Wittgensteins Sätze im TLP aber wirklich Unsinn sind, dann können sie nichts kommunizieren. Wir würden die Leiter gar nicht erst herauf kommen.

Das vierte Problem ist, dass Argumentation nur möglich ist, wenn das Argument eine innere logische Struktur hat. Das lässt sich bei deduktiven Argumenten besonders leicht erkennen. Solche Argumente sind in Diskussionen zulässig, weil ihre Konklusionen logisch aus ihren Prämissen folgen. Wittgenstein zufolge unterscheidet sich aber Unsinn von Sinnvollem und Sinnlosem gerade dadurch, dass er keine solche logische Struktur hat. Damit können mit unsinnigen Sätzen keine Argumente formuliert werden. Wir sollten philosophische Positionen aber nur akzeptieren, wenn gute Argumente für sie sprechen. Daher sollten wir Wittgensteins Position zurückweisen.

Wir haben also vier Probleme:

\begin{enumerate}
  \item Das Problem der Selbstwiderlegung des Philosophie-Skeptizismus
  \item Das Problem der Grenzen des Zeigbaren
  \item Das Problem der Kommunikation des TLP
  \item Das Problem der argumentativen Struktur des TLP
\end{enumerate}

\section{Conants Interpretation}

In Conants Interpretation sind die Ausdrücke \enquote{Unsinn} und \enquote{Erläuterung}	zentral. In der Literatur findet eine Debatte zwischen substanziellen und resoluten Lesarten des TLP statt, die sich Conant zufolge vor allem darum dreht, wie diese Ausdrücke zu interpretieren sind.\footnote{Conant nennt seine eigene Position und sein Unsinns- und Erläuterungsverständnis nicht \enquote{resolute}, sondern \enquote{austere} (Deutsch: streng, asketisch, karg, schmucklos.) Da \enquote{resolute} sich aber als Kategorie herausgestellt hat, durch die Lesarten wie seine in der Literatur allgemein bezeichnet werden, weiche ich hier von seiner Terminologie ab.} Nach Conant soll uns der TLP zuerst zu einem substanziellen Verständnis dieser Ausdrücke verführen, um uns später den Teppich unter den Füßen wegzuziehen, so dass wir zu einem resoluten Verständnis kommen. Ich werde also zunächst Conants Position umreißen, indem ich sein Verständnis von substanziellem und resolutem \enquote{Unsinn} und entsprechender \enquote{Erläuterung} rekonstruiere.

Ein substanzielles Verständnis von Unsinn unterscheidet zwischen \emph{substanziellem-} und \emph{einfach-nur-Unsinn}. Diese Unterscheidung diskutiert Conant mithilfe von Wittgensteins Unterscheidung zwischen Zeichen und Symbol. Ein bloßes Zeichen ist noch uninterpretiert, es hat keinen Sinn, keine Bedeutung und ist nichts weiter als ein Tintenklecks, Geräusch, Aufleuchten von Pixeln, etc. Ein Symbol ist mehr, es charakterisiert den Sinn eines Satzes. So handelt es sich zum Beispiel bei \enquote{ist} in den Sätzen \enquote{Bruce Wayne ist Batman.} und \enquote{Bruce Wayne ist Multimilliardär.} um dasselbe Zeichen aber um unterschiedliche Symbole. Im ersten Satz dient es als Kopula, im zweiten zum Ausdruck von Identität.\autocite[Vgl.][3.31-3.3442]{wittgenstein2016} Nach einem resoluten Verständnis gibt es auschließlich \emph{einfach-nur-Unsinn}. Substanzieller Unsinn sind Sätze, die unsinnig sind, weil in ihnen Symbole auf eine Art und Weise kombiniert werden, die die logische Syntax für solche Symbole nicht erlaubt. \emph{Einfach-nur-Unsinn} sind bloße sprachliche Zeichen, die weder einzeln noch gemeinsam Sinn oder Bedeutung haben. Der Unaussprechbarkeits-Lesart, einer besonderen Spielart der substanziellen Lesart, zufolge, drücken substanziell unsinnige Sätze so einen Gedanken aus, der in der unerlaubten Kombination der Bedeutungen der Symbole besteht, sich aber nicht aussprechen lässt.\autocite[Für Conants Diskussion der Lesarten mithilfe von Wittgensteins Unterscheidung zwischen Zeichen und Symbol vgl.][400-401]{conant2002}

Nach einem substanziellem Verständnis von Erläuterung erläutern die Sätze des TLP, indem sie einen solchen unaussprechlichen Gedanken kommunizieren. Nach einer resoluten Lesart erläutern sie, indem sie aufzeigen, dass sie selbst \emph{einfach-nur-Unsinn} sind.\autocite[Für die beiden Verständnisse von Unsinn und Erläuterung vgl.][380-381]{conant2002}

% TODO Die dialektische Vorstufe

So können wir auch die naive Rekonstruktion, die ich oben formuliert habe, nach Conant einordnen. Wittgenstein schreibt unsinnige Sätze wie \enquote{Der Sachverhalt ist eine Verbindung von Gegenständen (Sachen, Dingen).}\autocite[][2.01]{wittgenstein2016} Durch diese Sätze sollen wir erst annehmen, dass sie trotz ihrer Unsinnigkeit eine Einsicht kommunizieren, die unaussprechlich ist. Dass Sachverhalte also wirklich Verbindungen von Gegenständen sind, sich das aber nicht sagen lässt, weil die Symbole \enquote{Sachverhalt} und \enquote{Gegenstand} so kombiniert werden, wie es die logische Syntax verbietet. Später sollen wir dann merken, dass es gar keine unaussprechliche Einsicht zu kommunizieren gab, sondern Wittgensteins Sätze von Anfang an \emph{einfach-nur-Unsinn} waren.

% TODO Logische Kategorien übergreifende Äquivokation

So weit zu Conants Position. Kommen wir jetzt zu seiner Begründung. Nach Conant setzt sich der TLP mit einer Spannung in Freges Philosophie auseinander. Diese Spannung besteht zwischen der Einsicht, dass sich die Begriffsschrift nicht mit den Mitteln der Begriffsschrift schildern lässt auf der einen, und dem Kontextprinzip auf der andereren Seite.

Denn um zu vertreten, dass sich die Begriffschrift nicht mit den Mitteln der Begriffsschrift schildern lässt, muss Frege, so Conant, ein substanzielles Verständnis von Unsinn und Erläuterung vertreten. Schließlich ist Conant zufolge sinnvolle Sprache nach Frege nur im Rahmen der Begriffsschrift möglich. So etwas wie Freges Unterschied zwischen Begriff und Gegenstand lässt sich aber nicht durch die Begriffsschrift thematisieren. Denn in einem Satz wie \enquote{Der Begriff \emph{Primzahl} ist ungesättigt.} funktioniert \emph{Prinzahl} eben nicht als ungesättigter Begriff. Frege muss, so Conant, also annehmen, dass er seine Begriffsschrift durch unsinnige Äußerungen erläutert, die unaussprechliche Gedanken kommunizieren.\autocite[Zu Freges Einsicht, dass sich die Begriffsschrift nicht mit den Mitteln der Begriffsschrift erläutern lässt, vgl.][12-13]{frege1892}

Andererseits drängt das Kontextprinzip, so Conant, zu einem resoluten Verständnis von Unsinn. Es besagt: \enquote{nach der Bedeutung der Wörter muss im Satzzusammenhange, nicht in ihrer Vereinzelung gefragt werden}.\autocite[][]{frege1884} Wittgenstein wiederholt das in eigenen Worten als \enquote{Nur der Satz hat Sinn; nur im Zusammenhange des Satzes hat ein Name Bedeutung.}\footnote{\cite{wittgenstein1922}, 3.3.} Conant versteht das so, dass wir Bestandteile eines Satzes nur dann als Symbole identifizieren können, wenn wir den gesamten Satz sinnvoll interpretieren. Da das bei unsinnigen Sätzen nie geht, sind sie nichts weiter als Zeichenketten, es gibt also nur \emph{einfach-nur-Unsinn}. Das Problem mit Sätzen wie \enquote{Sokrates ist identisch.}, ist nicht, dass sie unerlaubte Symbole kombinieren, sondern dass sie logische Kategorien übergreifende Äquivokationen sind. Von all den Interpretationen, die wir für den Satz aufstellen könnten, gibt es keine, die plausibel für alle seine Komponenten ist.

% TODO Die große Einsicht

Nach Conant löst Wittgenstein diese Spannung hin zu einem resoluten Verständnis von Unsinn und Erläuterung auf. Anstatt unaussprechliche Einsichten zu kommunizieren, sollen die Sätze des TLP dann eine besondere Erfahrung in der Leser*in erzeugen. Zuerst wird sie zu einer Theorie von Sprache und Welt verlockt, nach der die Sprache die Welt abbildet, dieses Verhältnis sich aber sprachlich nicht ausdrücken lässt. Die Leser*in wird also erst von der substanziellen Konzeption von Unsinn überzeugt, nach der es unaussprechbare Einsichten gibt. Hier erweckt der TLP den Anschein, dass er aus Argumenten besteht, die Positionen stützen. Dann wird der Unsinn, der die Leser*in zu dieser Konzeption verleitete, aber als logische Kategorien übergreifende Äquivokation aufgezeigt. Dabei wird die Illusion von Sinn, zu der sie verleitet wurde, zerbrochen: Mit dem Unsinn war überhaupt nichts gemeint. So wirft die LeserIn die Leiter um. Sie erkennt den gesamten TLP als Unsinn. Dennoch hat sie sich aber an einer Praxis beteiligt und dabei eine besondere Erfahrung gemacht, die sie verändert hat. Denn jetzt, da sie zuerst getäuscht und dann aufgeklärt wurde, wird sie nicht mehr in die erstere Täuschung zurückfallen.\autocite[Vgl.][421-424]{conant2002}

% TODO Inwiefern das Leiterprobleme löst

\textbf{Substanzielle Lesart von Unsinn und Erläuterung und resolute Alternative}

Nach Conant sollen die unsinnigen Sätze des TLP eine bestimmte Art von Erfahrung in der LeserIn hervorbringen. Zuerst wird sie zu einer Theorie von Sprache und Welt verlockt, nach der die Sprache die Welt abbildet, dieses Verhältnis sich aber sprachlich nicht ausdrücken lässt. Die LeserIn wird also erst von der substanziellen Konzeption von Unsinn überzeugt, nach der es unaussprechbare Einsichten gibt. Hier erweckt der TLP den Anschein, dass er aus Argumenten besteht, die Positionen stützen. Dann wird der Unsinn, der die LeserIn zu dieser Konzeption verleitete, aber als logische Kategorien übergreifende Äquivokation aufgezeigt. Dabei wird die Illusion von Sinn, zu der sie verleitet wurde, zerbrochen: Mit dem Unsinn war überhaupt nichts gemeint. So wirft die LeserIn die Leiter um. Sie erkennt den gesamten TLP als Unsinn. Dennoch hat sie sich aber an einer Praxis beteiligt und dabei eine besondere Erfahrung gemacht, die sie verändert hat. Denn jetzt, da sie zuerst getäuscht und dann aufgeklärt wurde, wird sie nicht mehr in die erstere Täuschung zurückfallen.\autocite[Vgl.][421-424]{conant2002}

\section{Sind die Probleme damit gelöst?}

Gehen wir also die Probleme durch und prüfen, ob Conants Interpretation sie lösen kann. Angefangen mit dem Problem der Selbstwiderlegung des Philosophie-Skeptizismus. Hier beansprucht Conant keine Lösung. Der TLP soll ihm zufolge größtenteils selbstwiderlegend sein. All die klassischen Gedanken, also dass Sätze sich wie Bilder auf die Welt beziehen, dass die Welt aus Tatsachen besteht, der Sinn eines Satzes seine Wahrheitsbedingungen sind, etc. weist Wittgenstein, so Conant, als unsinnig zurück.

\section{Fazit}

% Literaturverzeichnis

% \nocite{hab_euch_benutzt_aber_nicht_zitiert}

\clearpage
\printbibliography

% Eigenständigkeitserklärung
\clearpage

\section*{Eigenständigkeitserklärung}

Hiermit versichere ich, dass ich diese Arbeit mit Ausnahme von dieser Erklärung ausschließlich mit den angegebenen Hilfsmitteln verfasst habe. Alle Passagen, die ich wörtlich oder sinngemäß aus Büchern, Artikeln oder anderen Quellen übernommen habe, habe ich als Zitate kenntlich gemacht. Darüber hinaus versichere ich, dass diese Arbeit nicht schon Gegenstand eines anderen Prüfungsverfahrens gewesen ist.

\end{document}
